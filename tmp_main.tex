\documentclass[UTF8]{ctexart}
\setcounter{secnumdepth}{3} % 设置编号的深度
\renewcommand\appendix{\setcounter{secnumdepth}{-2}}
\usepackage[a4paper, margin=2cm]{geometry}
\usepackage{chapterbib}
\usepackage{amsmath}
\usepackage{caption}
\usepackage{subfig}
\usepackage{booktabs}
\usepackage{amsbsy}
\usepackage{listings}
\usepackage{nomencl}
\makenomenclature

\usepackage[numbers]{gbt7714}
% \captionsetup[figure]{labelfont={bf},labelsep=quad} % 修改图片默认标题
% \captionsetup[table]{labelfont={bf},labelsep=quad} % 修改图片默认标题
\captionsetup[figure]{labelsep=quad} % 修改图片默认标题
\captionsetup[table]{labelsep=quad} % 修改图片默认标题
\numberwithin{equation}{section}
\numberwithin{figure}{section}
\numberwithin{table}{section}
\usepackage{amssymb}
\usepackage{graphicx}
\DeclareMathOperator*{\rint}{\ThisStyle{\rotatebox{15}{$\SavedStyle\!\int\!$}}}
\usepackage{scalerel}

\usepackage[dvipsnames]{xcolor} %xcolor 包要放在 tikz 包前面,否则会报错
\colorlet{mypurple}{blue!66} % 自定义紫色

\usepackage{tikz}
\newcommand*\circled[1]{\tikz[baseline=(char.base)]{
            \node[shape=circle,draw,inner sep=2pt] (char) {#1};}} % 用于制作 ①

\usepackage[colorlinks,linkcolor=black,anchorcolor=orange,citecolor=black,urlcolor=black]{hyperref} %设置链接颜色
\usepackage{pdfpages}
\usepackage[noend]{algpseudocode}
\usepackage{algorithmicx,algorithm}

% \newcommand{\kw}[1]{\textcolor{mypurple}{#1}} % 设置关键字的颜色
\newcommand{\ks}[1]{\textbf{#1}} % 设置关键句子加粗
\newcommand{\kw}[1]{\textbf{#1}} % 设置关键字加粗
\renewcommand\bibname{参考文献}
% \pagestyle{plain}
\usepackage{fancyhdr}
\usepackage{tcolorbox}
\usepackage{listings} % 代码包
\usepackage{enumitem}
\setlist{nosep}

\renewcommand{\nomname}{主要符号表} 
\renewcommand{\nomlabel}[1]{\hfil #1\hfil}
% \renewcommand{\nomlabel}[1]{\centering #1}

\newcommand{\figref}[1]{图~\ref{#1}}
\renewcommand{\eqref}[1]{式~(\ref{#1})}
\newcommand{\tabref}[1]{表~\ref{#1}}%
\newcommand{\coderef}[1]{代码~\ref{#1}}

\newcommand{\upcite}[1]{\textsuperscript{\cite{#1}}} % 实现右上角引用
% \newcommand{\secref}[1]{Sec.~\ref{#1}}
% \renewcommand{\eqref}[1]{Equation~(\ref{#1})}

\newcommand{\code}[1]{{\textcolor{magenta}{#1}}}
% \usepackage{hyperref}
\pagestyle{fancy}
% \fancyhead[L]{→\_→}
\fancyhead[L]{}
% \fancyhead[R]{←\_←}
\fancyhead[R]{}
% \fancyhead[C]{https://github.com/datawhalechina/easy-rl}
\ctexset{
    section = {
        name = {第,章}
    }
}

\lstset{
    basicstyle          =   \sffamily,          % 基本代码风格
    keywordstyle        =   \bfseries,          % 关键字风格
    commentstyle        =   \rmfamily\itshape,  % 注释的风格,斜体
    stringstyle         =   \ttfamily,  % 字符串风格
    flexiblecolumns,                % 别问为什么,加上这个
    % numbers             =   left,   % 行号的位置在左边
    showspaces          =   false,  % 是否显示空格,显示了有点乱,所以不现实了
    numberstyle         =   \zihao{-5}\ttfamily,    % 行号的样式,小五号,tt等宽字体
    showstringspaces    =   false,
    captionpos          =   t,      % 这段代码的名字所呈现的位置,t指的是top上面
    frame               =   lrtb,   % 显示边框
    commentstyle        =   \color{ForestGreen}\ttfamily, % 设置注释的颜色以及字体
}

\lstdefinestyle{Python}{
    language        =   Python, % 语言选Python
    basicstyle      =   \zihao{-5}\ttfamily,
    numberstyle     =   \zihao{-5}\ttfamily,
    keywordstyle    =   \color{blue},
    keywordstyle    =   [2] \color{Bittersweet},
    stringstyle     =   \color{magenta},
    commentstyle    =   \color{ForestGreen}\ttfamily,
    breaklines      =   true,   % 自动换行,建议不要写太长的行
    columns         =   fixed,  % 如果不加这一句,字间距就不固定,很丑,必须加
    basewidth       =   0.5em,
}


\title{强化学习:原理与实践 \\[0.4cm] Easy RL}
\author{Datawhale}
\date{2021年2月}

\begin{document}
    % \includepdf[fitpaper=true]{res/cover.pdf}
    % \usepackage[a4paper, left=3.17cm, right=3.17cm, top=2.54cm, bottom=2.54cm]{geometry}
% \usepackage[T1]{fontenc}
% \usepackage{mathptmx}
% \usepackage{amsmath}
% \usepackage{amsfonts}
% \usepackage{chemformula}
% \usepackage{cite}
% \usepackage[colorlinks, linkcolor=black, anchorcolor=black, citecolor=black]{hyperref}
% \usepackage{graphicx}
% \setlength{\parskip}{0.5em}
% \title{Place Your Project Title at Here}
% \author{\textup{Sanlinc}}

\begin{titlepage}
	\newcommand{\HRule}{\rule{\linewidth}{0.5mm}}
	\includegraphics[width=2cm]{res/logo_new.png}\\[1cm] 
	\center 
	\quad\\[1.5cm]
	% \textsl{\Large University of Chinese Academy of Sciences }\\[0.5cm] 
	
	\makeatletter
	\HRule \\[0.4cm]
	{ \huge \bfseries \@title}\\[0.4cm] 
	\HRule \\[1.5cm]
    % \Large \url{https://github.com/datawhalechina/easy-rl}\\[0.5cm] 
	\begin{minipage}{0.4\textwidth}
		\begin{flushleft} \large
			% \emph{Author:}\\
			% \@author 
		\end{flushleft}
	\end{minipage}
	~
	\begin{minipage}{0.4\textwidth}
		\begin{flushright} \large
			% \emph{Supervisor:} \\
			% \textup{Prof Yang}
		\end{flushright}
	\end{minipage}\\[3cm]
	\makeatother
	% {\large An Assignment submitted for the UCAS:}\\[0.5cm]
    \vfill  % 为了制造Whitespace 
	% {\large Datawhale强化学习小组~编著}\\[0.5cm]
    % {\Large Qi Wang~~~David Young~~~John Jim~编著}\\[0.5cm]
    % {\Large \href{https://github.com/qiwang067}{Qi Wang}~~~\href{https://github.com/yyysjz1997}{Yiyuan Yang}~~~\href{https://github.com/JohnJim0816}{Ji Jiang} ~编著}\\[0.5cm]
	{\Large 王琦~~~杨毅远~~~江季 ~编著}\\[0.5cm]
    {\Large  版本:1.0.2}\\[0.5cm]
	{\Large \today}\\[2cm] 
	
\end{titlepage}


    \thispagestyle{empty}
    \clearpage
    
    \section*{内容提要}
    强化学习作为机器学习及人工智能领域的一种重要方法,在游戏、自动驾驶、机器人路线规划等领域得到了广泛的应用。
    本书结合了李宏毅老师的《深度强化学习》、周博磊老师的《强化学习纲要》、李科浇老师的《百度强化学习》公开课的精华内容,在理论严谨的基础上深入浅出地介绍马尔可夫决策过程、蒙特卡洛方法、时序差分方法、Sarsa、Q学习等传统强化学习算法,以及策略梯度、近端策略优化、深度Q网络、深度确定性策略梯度等常见深度强化学习算法的基本概念和方法,并以大量生动有趣的例子帮助读者理解强化学习问题的建模过程以及核心算法的细节。此外,本书还提供较为全面的习题解答以及Python代码实现,可以让读者进行端到端、从理论到完全实践的全生态学习,充分掌握强化学习算法的原理并能进行实战。

    本书适合对强化学习感兴趣的读者阅读,也可以作为相关课程的配套教材。


    \clearpage
    \section*{前言}
    \renewcommand{\abstractname}{}
    % \begin{abstract}


    这是一本面向中文读者的强化学习教科书,为了使尽可能多的读者通过本书对强化学习有所了解,笔者试图尽可能少地使用数学知识,所涉及的公式都有详细的推导过程。本书适合相关专业的本科生和研究生,以及具有类似背景的对强化学习感兴趣的人士。

    全书共13章,大体上可分为2个部分:第1部分包括第1 \~{} 3章,介绍强化学习基础知识以及传统强化学习算法;第2部分包括第4 \~{} 13章,介绍深度强化学习算法以及常见问题的解决方法。第2部分各章相对独立,读者可根据自己的兴趣和时间情况选择阅读。
        
    书中大部分章配有习题和面试题,其可以帮助读者巩固知识。读者遇到某个不熟悉的概念时,还可以通过“关键词”部分来快速定位并掌握该概念。
        
    笔者以为,强化学习是一个理论与实践相结合的学科,读者不仅要理解其算法背后的一些数学原理,还要通过上机实践来实现算法。本书配有对应的 Python 代码实现,可以让读者通过动手实现各种经典的强化学习算法,充分掌握强化学习算法的原理。
        
    书中主要内容源于李宏毅老师的《深度强化学习》、周博磊老师的《强化学习纲要》以及李科浇老师的《百度强化学习》公开课。3位老师的强化学习公开课深入浅出、生动有趣,是强化学习的经典学习材料。感谢李宏毅、周博磊、李科浇3位老师的授权,使本书得以出版,能够造福更多对强化学习感兴趣的读者。
        
    本书由开源组织 Datawhale 的成员采用开源协作的方式完成,共历时一年有余,参与者包括3位编著者(笔者、杨毅远和江季)和两位 Datawhale 的小伙伴(谢文睿和马燕鹏)。在本书写作和出版过程中,人民邮电出版社的责任编辑郭媛给予了很多帮助,在此特向她致谢。
        
    强化学习发展迅速,笔者水平有限,书中难免有疏漏和表述不当的地方,还望各位读者批评指正。

\rightline{王琦~~~~~~~~~~~~~~~~~~}
\rightline{\today~~~~~~~~~~}

        % \subsection*{使用说明}
        % \begin{itemize}
        %     \item 
        %     第 4 章到第 11 章为\href{http://speech.ee.ntu.edu.tw/~tlkagk/courses_MLDS18.html}{李宏毅《深度强化学习》}的部分;
        %     \item 
        %     第 1 章和第 2 章根据\href{https://github.com/zhoubolei/introRL}{《强化学习纲要》}整理而来;
        %     \item 
        %     第 3 章和第 12 章根据\href{https://aistudio.baidu.com/aistudio/education/group/info/1335}{《百度强化学习》}整理而来。
        % \end{itemize}
        % \textbf{在线阅读地址:}\url{https://datawhalechina.github.io/easy-rl/}(内容实时更新)\\
        % \textbf{最新版PDF获取地址:}\url{https://github.com/datawhalechina/easy-rl/releases}
        % \subsection*{编委会}
        % \noindent
        % % \textbf{编委:}\href{https://github.com/qiwang067}{Qi Wang}、\href{https://github.com/yyysjz1997}{Yiyuan Yang}、\href{https://github.com/JohnJim0816}{Ji Jiang} 
        % \textbf{编委:}王琦、杨毅远、江季
        % % \textbf{编委:}jbb0523、juxiao、Majingmin、MrBigFan、shanry、Ye980226
        % \subsection*{致谢}
        % 特别感谢 \href{https://github.com/Sm1les}{Sm1les}、\href{https://github.com/LSGOMYP}{LSGOMYP} 对本项目的帮助与支持。
        % \noindent
        % \begin{center}
        % \vspace{1em}
        % 扫描下方二维码,然后回复关键词“\textbf{强化学习}”,即可加入“Easy-RL读者交流群”\\
        % \begin{figure}[hb]
        % \centering
        % \includegraphics[scale=0.2]{res/qrcode.jpeg}
        % \end{figure}
        % Datawhale\\一个专注于AI领域的开源组织\\
        % \vspace{2em}
        % \textbf{版权声明}\\本作品采用\href{http://creativecommons.org/licenses/by-nc-sa/4.0/}{\textbf{知识共享署名-非商业性使用-相同方式共享 4.0 国际许可协议}}进行许可。
        % \end{center}
    % \end{abstract}
    \thispagestyle{empty}
    \clearpage
    \setlength{\nomitemsep}{0.8cm}

\nomenclature[01]{$a$}{标量}
\nomenclature[02]{$\boldsymbol{a}$}{向量}
\nomenclature[03]{$\boldsymbol{A}$}{矩阵}
\nomenclature[04]{$\mathbb{R}$}{实数集}
\nomenclature[05]{$\underset{a}{\arg \max } f(a)$}{$f(a)$取最大值时 $a$ 的值}
\nomenclature[06]{$s$}{状态}
\nomenclature[07]{$a$}{动作}
\nomenclature[08]{$r$}{奖励}
\nomenclature[09]{$\pi$}{策略}
\nomenclature[10]{$\pi(s)$}{根据确定性策略 $\pi$ 在状态 $s$ 选取的动作}
\nomenclature[11]{$\pi(a|s)$}{根据随机性策略 $\pi$ 在状态 $s$ 选取的动作 $a$ 的概率}
\nomenclature[12]{$\gamma$}{折扣因子}
\nomenclature[13]{$\tau$}{轨迹}
\nomenclature[14]{$V_{\pi}(s)$}{状态 $s$ 在策略 $\pi$ 下的价值}
\nomenclature[15]{$Q_{\pi}(s,a)$}{状态 $s$ 在策略 $\pi$ 下采取动作 $a$ 的价值}
\nomenclature[16]{$G_{t}$}{时刻 $t$ 时的回报}
\nomenclature[17]{$\pi_{\theta}$}{参数$\theta$ 对应的策略}
\nomenclature[18]{$J({\theta})$}{策略$\pi_{\theta}$的性能度量}

\printnomenclature[4in]





    \tableofcontents % 生成目录
    % \thispagestyle{empty}
    \clearpage

    % \section{动态规划}
前面我们详细讲述了马尔可夫决策过程(MDP)的相关原理和性质,实际上 \kw{MDP 是强化学习的基础模型}
本节开始讲强化学习算法中最基础的算法之一,即动态规划。
\subsection{策略迭代}
\subsection{价值迭代}
\subsection{实战:策略迭代算法}
\subsection{实战:价值迭代算法}
\subsection{关键词}
\subsection{习题}
\subsection{面试题}
\subsection{本章小结}
    % \include{ch2}
    % \section{表格型方法}
\subsection{蒙特卡洛方法}
\subsection{时序差分方法}
\subsection{ Q-learning 算法}
\subsection{ Sarsa 算法}
\subsection{ Dyna-Q 算法}
\subsubsection{有模型与免模型}
\subsection{实战:Q-learning 算法}
\subsection{实战:Sarsa 算法}
\subsection{实战:Dyna-Q 算法}
\subsection{关键词}
\subsection{习题}
\subsection{面试题}
\subsection{本章小结}
    % \include{ch4}
    % \section{ DQN 算法进阶}
\subsection{ Double DQN 算法}

\subsection{ Dueling DQN 算法}

\subsection{ PER DQN 算法}
\subsection{ HER DQN 算法}

\subsection{ Noisy DQN 算法}
\subsection{ QR DQN 算法}
\subsection{ Rainbow DQN 算法}


\subsection{实战:Double DQN 算法}

\subsection{实战:Dueling DQN 算法}

\subsection{实战:PER DQN 算法}

\subsection{实战:HER DQN 算法}
\subsection{实战:Noisy DQN 算法}
\subsection{实战:QR DQN 算法}
\subsection{实战:Rainbow DQN 算法}

\subsection{关键词}
\subsection{习题}
\subsection{面试题}
\subsection{本章小结}
    \section{马尔可夫决策过程}

\figref{fig:rl_pic} 介绍了强化学习中智能体与环境之间的交互过程,智能体观测到环境的状态后,它会采取动作,然后环境根据智能体采取的动作进入到下一个状态,并反馈出对应的奖励信号。举一个踢足球的例子,假设我们一开始什么也不会,但想要学习罚点球,也就是在点球点处将球踢进球门里去,每次练习开始我们观测到球和球门的位置,并尝试调整角度将球踢出去。踢出去之后球的位置会发生变化,根据球和球门的位置我们可以接收到反馈,比如球是进球门了还是踢飞了。接着我们将球重新放回点球点处继续练习,并根据上一次收到的反馈调整我们的射门动作,如此循环交互,直到学会将球踢进球门为止。在这个过程中,我们人就相当于是智能体,球和球门的位置就是人观测到的状态,将球踢出去就是一个动作,而球、球门和球场组成整个环境,每次采取动作后接收到的反馈可以数值化成一个奖励信号,进球门了就是正的奖励,踢飞了或者没踢进就是负的奖励,也就是惩罚。生活中的很多问题都可以用这样一段持续的交互过程来描述,而这个交互过程在数学上可以用马尔可夫决策过程(Markov decision process,MDP),而用于解决 MDP 问题的方法都可以被认为是强化学习方法,因此\kw{ MDP 是强化学习的基本问题模型之一}。换句话说,要想用强化学习算法解决某个实际问题,我们必须首先将这个问题描述或者建模成一个马尔可夫决策过程。顺便说一句,之所以说MDP 是强化学习的基本问题模型之一,这里之一的意思是在更多复杂的情况中比如多智能体环境我们需要把问题建模成一个 MDP 的衍生版本,比如 部分可观测马尔可夫决策过程(partially observable Markov decision processes,POMDP) 以及马尔可夫博弈等等,但广义上来说都属于马尔可夫决策过程,具体在后面讲述相关内容时再详细展开。


\begin{figure}[hbt]
  \centering
  \includegraphics[width=0.5\linewidth]{ch2/figs/rl_pic.png}
  \caption{智能体与环境之间的交互}
  \label{fig:rl_pic}
\end{figure}


本章将介绍马尔可夫决策过程(Markov decision process,MDP)。在介绍马尔可夫决策过程之前,我们先介绍它的简化版本:马尔可夫过程(Markov process,MP)以及马尔可夫奖励过程(Markov reward process,MRP)。通过这两种过程的铺垫,我们可以更容易理解马尔可夫决策过程。
\subsection{马尔可夫过程}
\subsubsection{马尔可夫性质}
\kw{马尔可夫性质(Markov property)}是概率论中的一个概念,指的是当一个随机过程在给定当前状态及所有过去状态情况下,其未来状态的条件概率分布仅依赖于当前状态。换句话说,在给定当前状态下,它与过去状态(即该过程的历史路径)是条件独立的。

以离散随机过程为例,假设随机变量 $X_0,X_1,\cdots,X_T$构成一个随机过程。这些随机变量的所有可能取值的集合被称为状态空间(state space)。如果 $X_{t+1}$ 对于过去状态的条件概率分布仅是 $X_t$ 的一个函数,则
\begin{equation}
  \label{eq:}
  p\left(X_{t+1}=x_{t+1} \mid X_{0:t}=x_{0: t}\right)=p\left(X_{t+1}=x_{t+1} \mid X_{t}=x_{t}\right)
\end{equation}
其中,$X_{0:t}$ 表示变量集合 $X_{0}, X_{1}, \cdots, X_{t}$,$x_{0: t}$ 为在状态空间中的状态序列 $x_{0}, x_{1}, \cdots, x_{t}$。

马尔可夫性质是所有马尔可夫过程的基础。这种性质看似深奥,其实在我们日常生活中也司空见惯,比如一天内,我晚餐的食量由午餐的时间和摄入直接决定,而不由早餐的时间和摄入间接决定(因为早餐摄入的食物在晚餐前就已经消化掉了),这样一来一日三餐的摄入就可以简单看作一个马尔可夫过程。

\subsubsection{马尔可夫链}
% 如果一个状态转移是符合马尔可夫的,也就是满足条件:
马尔可夫过程是一组具有马尔可夫性质的随机变量序列 $s_1,\cdots,s_t$,其中下一个时刻的状态$s_{t+1}$只取决于当前状态 $s_t$。我们设状态的历史为 $h_{t}=\left\{s_{1}, s_{2}, s_{3}, \ldots, s_{t}\right\}$($h_t$ 包含了之前的所有状态),则马尔可夫过程满足条件:
\begin{equation}
  \label{eq:}
  p\left(s_{t+1} \mid s_{t}\right) =p\left(s_{t+1} \mid h_{t}\right)
\end{equation}
从当前 $s_t$ 转移到 $s_{t+1}$,它是直接就等于它之前所有的状态转移到 $s_{t+1}$。

离散时间的马尔可夫过程也称为\kw{马尔可夫链(Markov chain)}。马尔可夫链是最简单的马尔可夫过程,其状态是有限的。例如,\figref{fig:mp_example} 里面有4个状态,这4个状态在 $s_1,s_2,s_3,s_4$ 之间互相转移。比如从 $s_1$ 开始,$s_1$ 有 0.1 的概率继续存留在 $s_1$ 状态,有 0.2 的概率转移到 $s_2$,有 0.7 的概率转移到 $s_4$ 。如果 $s_4$ 是我们的当前状态,它有 0.3 的概率转移到 $s_2$,有 0.2 的概率转移到 $s_3$,有 0.5 的概率留在当前状态。

\begin{figure}[htb]
  \centering
  \includegraphics[width=0.3\linewidth]{ch2/figs/m_chain_example.png}
  \caption{马尔可夫链示例}
  \label{fig:mp_example}
\end{figure}

我们可以用\kw{状态转移矩阵(state transition matrix)}$\boldsymbol{P}$ 来描述状态转移 $p\left(s_{t+1}=s^{\prime} \mid s_{t}=s\right)$:
\begin{equation}
  \boldsymbol{P}=\left(\begin{array}{cccc}
    p\left(s_{1} \mid s_{1}\right) & p\left(s_{2} \mid s_{1}\right) & \ldots & p\left(s_{N} \mid s_{1}\right) \\
    p\left(s_{1} \mid s_{2}\right) & p\left(s_{2} \mid s_{2}\right) & \ldots & p\left(s_{N} \mid s_{2}\right) \\
    \vdots & \vdots & \ddots & \vdots \\
    p\left(s_{1} \mid s_{N}\right) & p\left(s_{2} \mid s_{N}\right) & \ldots & p\left(s_{N} \mid s_{N}\right)
    \end{array}\right)
  \label{eq:1}
\end{equation}
状态转移矩阵类似于条件概率(conditional probability),它表示当我们知道当前我们在状态 $s_t$ 时,到达下面所有状态的概率。所以它的每一行描述的是从一个节点到达所有其他节点的概率。

\subsection{马尔可夫奖励过程} 

\kw{马尔可夫奖励过程(Markov reward process, MRP)}是马尔可夫链加上奖励函数。在马尔可夫奖励过程中,状态转移矩阵和状态都与马尔可夫链一样,只是多了\kw{奖励函数(reward function)}。奖励函数 $R$ 是一个期望,表示当我们到达某一个状态的时候,可以获得多大的奖励。这里另外定义了折扣因子 $\gamma$ 。如果状态数是有限的,那么 $R$ 可以是一个向量。

\subsubsection{回报与价值函数}

这里我们进一步定义一些概念。\kw{范围(horizon)} 是指一个回合的长度(每个回合最大的时间步数),它是由有限个步数决定的。
\kw{回报(return)}可以定义为奖励的逐步叠加,假设时刻$t$后的奖励序列为$r_{t+1},r_{t+2},r_{t+3},\cdots$,则回报为
\begin{equation}
  G_{t}=r_{t+1}+\gamma r_{t+2}+\gamma^{2} r_{t+3}+\gamma^{3} r_{t+4}+\ldots+\gamma^{T-t-1} r_{T}
  \label{eq:}
\end{equation}
其中,$T$是最终时刻,$\gamma$ 是折扣因子,越往后得到的奖励,折扣越多。这说明我们更希望得到现有的奖励,对未来的奖励要打折扣。
当我们有了回报之后,就可以定义状态的价值了,就是\kw{状态价值函数(state-value function)}。对于马尔可夫奖励过程,状态价值函数被定义成回报的期望,即
\begin{equation}
  \begin{aligned}
    V^{t}(s) &=\mathbb{E}\left[G_{t} \mid s_{t}=s\right] \\
    &=\mathbb{E}\left[r_{t+1}+\gamma r_{t+2}+\gamma^{2} r_{t+3}+\ldots+\gamma^{T-t-1} r_{T} \mid s_{t}=s\right]
    \end{aligned}
  \label{eq:}
\end{equation}
其中,$G_t$ 是之前定义的\kw{折扣回报(discounted return)}。我们对$G_t$取了一个期望,期望就是从这个状态开始,我们可能获得多大的价值。所以期望也可以看成未来可能获得奖励的当前价值的表现,就是当我们进入某一个状态后,我们现在有多大的价值。

我们使用折扣因子的原因如下。第一,有些马尔可夫过程是带环的,它并不会终结,我们想避免无穷的奖励。第二,我们并不能建立完美的模拟环境的模型,我们对未来的评估不一定是准确的,我们不一定完全信任模型,因为这种不确定性,所以我们对未来的评估增加一个折扣。我们想把这个不确定性表示出来,希望尽可能快地得到奖励,而不是在未来某一个点得到奖励。
第三,如果奖励是有实际价值的,我们可能更希望立刻就得到奖励,而不是后面再得到奖励(现在的钱比以后的钱更有价值)。
最后,我们也更想得到即时奖励。有些时候可以把折扣因子设为 0($\gamma=0$),我们就只关注当前的奖励。
我们也可以把折扣因子设为 1($\gamma=1$),对未来的奖励并没有打折扣,未来获得的奖励与当前获得的奖励是一样的。
折扣因子可以作为强化学习智能体的一个超参数(hyperparameter)来进行调整,通过调整折扣因子,我们可以得到不同动作的智能体。

在马尔可夫奖励过程里面,我们如何计算价值呢?如\figref{fig:fig2.11} 所示,马尔可夫奖励过程依旧是状态转移,其奖励函数可以定义为:智能体进入第一个状态 $s_1$ 的时候会得到 5 的奖励,进入第七个状态 $s_7$ 的时候会得到 10 的奖励,进入其他状态都没有奖励。我们可以用向量来表示奖励函数,即

\begin{equation}
  \label{eq:}
  \boldsymbol{R}=[5,0,0,0,0,0,10]
\end{equation}

\begin{figure}[hbt]
  \centering
  \includegraphics[width=0.7\linewidth]{ch2/figs/2.11.png}
  \caption{马尔可夫奖励过程的例子}
  \label{fig:fig2.11}
\end{figure}

我们对 4 步的回合($\gamma=0.5$)来采样回报 $G$。
% \begin{enumerate}[label=\protect\circled{\arabic*}]

  (1)$s_{4}, s_{5}, s_{6}, s_{7} \text{的回报}: 0+0.5\times 0+0.25 \times 0+ 0.125\times 10=1.25$

  (2)$s_{4}, s_{3}, s_{2}, s_{1} \text{的回报}: 0+0.5 \times 0+0.25\times 0+0.125 \times 5=0.625$

  (3)$s_{4}, s_{5}, s_{6}, s_{6} \text{的回报}: 0+0.5\times 0 +0.25 \times 0+0.125 \times 0=0$
% \end{enumerate}

我们现在可以计算每一个轨迹得到的奖励,比如我们对轨迹 $s_4,s_5,s_6,s_7$ 的奖励进行计算,这里折扣因子是 0.5。
在 $s_4$ 的时候,奖励为0。
下一个状态 $s_5$ 的时候,因为我们已经到了下一步,所以要把 $s_5$ 进行折扣,$s_5$ 的奖励也是0。
然后是 $s_6$,奖励也是0,折扣因子应该是0.25。
到达 $s_7$ 后,我们获得了一个奖励,但是因为状态 $s_7$ 的奖励是未来才获得的奖励,所以我们要对之进行3次折扣。
最终这个轨迹的回报就是 1.25。类似地,我们可以得到其他轨迹的回报。

这里就引出了一个问题,当我们有了一些轨迹的实际回报时,怎么计算它的价值函数呢?比如我们想知道 $s_4$ 的价值,即当我们进入 $s_4$ 后,它的价值到底如何?一个可行的做法就是我们可以生成很多轨迹,然后把轨迹都叠加起来。比如我们可以从 $s_4$ 开始,采样生成很多轨迹,把这些轨迹的回报都计算出来,然后将其取平均值作为我们进入 $s_4$ 的价值。这其实是一种计算价值函数的办法,也就是通过蒙特卡洛(Monte Carlo,MC)采样的方法计算 $s_4$ 的价值。

\subsubsection{马尔可夫奖励过程的例子} 

如\figref{fig:mrp_example} 所示,如果我们在马尔可夫链上加上奖励,那么到达每个状态,我们都会获得一个奖励。我们可以设置对应的奖励,比如智能体到达状态 $s_1$时,可以获得 5 的奖励;到达 $s_7$ 的时候,可以得到 10 的奖励;到达其他状态没有任何奖励。
因为这里的状态是有限的,所以我们可以用向量 $\boldsymbol{R}=[5,0,0,0,0,0,10]$ 来表示奖励函数,$\boldsymbol{R}$表示每个状态的奖励大小。

我们通过一个形象的例子来理解马尔可夫奖励过程。我们把一艘纸船放到河流之中,它就会随着水流而流动,它自身是没有动力的。所以我们可以把马尔可夫奖励过程看成一个随波逐流的例子,当我们从某一个点开始的时候,纸船就会随着事先定义好的状态转移进行流动,它到达每个状态后,我们都有可能获得一些奖励。

\begin{figure}[hbt]
  \centering
  \includegraphics[width=0.5\linewidth]{ch2/figs/2.8}
  \caption{马尔可夫奖励过程的例子}
  \label{fig:mrp_example}
\end{figure}

\subsection{马尔可夫决策过程} 
相对于马尔可夫奖励过程,马尔可夫决策过程多了决策(决策是指动作),其他的定义与马尔可夫奖励过程的是类似的。此外,状态转移也多了一个条件,变成了$p\left(s_{t+1}=s^{\prime} \mid s_{t}=s,a_{t}=a\right)$。未来的状态不仅依赖于当前的状态,也依赖于在当前状态智能体采取的动作。马尔可夫决策过程满足条件:
\begin{equation}
  \label{eq:}
  p\left(s_{t+1} \mid s_{t}, a_{t}\right) =p\left(s_{t+1} \mid h_{t}, a_{t}\right)   
\end{equation}

对于奖励函数,它也多了一个当前的动作,变成了 $R\left(s_{t}=s, a_{t}=a\right)=\mathbb{E}\left[r_{t} \mid s_{t}=s, a_{t}=a\right]$。当前的状态以及采取的动作会决定智能体在当前可能得到的奖励多少。


\subsubsection{马尔可夫决策过程中的策略} 

策略定义了在某一个状态应该采取什么样的动作。知道当前状态后,我们可以把当前状态代入策略函数来得到一个概率,即 
\begin{equation}
  \pi(a \mid s)=p\left(a_{t}=a \mid s_{t}=s\right)
  \label{eq:}
\end{equation}
概率代表在所有可能的动作里面怎样采取行动,比如可能有 0.7 的概率往左走,有 0.3 的概率往右走,这是一个概率的表示。
另外策略也可能是确定的,它有可能直接输出一个值,或者直接告诉我们当前应该采取什么样的动作,而不是一个动作的概率。
假设概率函数是平稳的(stationary),不同时间点,我们采取的动作其实都是在对策略函数进行采样。

已知马尔可夫决策过程和策略 $\pi$,我们可以把马尔可夫决策过程转换成马尔可夫奖励过程。
在马尔可夫决策过程里面,状态转移函数 $P(s'|s,a)$ 基于它当前的状态以及它当前的动作。因为我们现在已知策略函数,也就是已知在每一个状态下,可能采取的动作的概率,所以我们就可以直接把动作进行加和,去掉 $a$,这样我们就可以得到对于马尔可夫奖励过程的转移,这里就没有动作,即
\begin{equation}
  P_{\pi}\left(s^{\prime} \mid s\right)=\sum_{a \in A} \pi(a \mid s) p\left(s^{\prime} \mid s, a\right)
  \label{eq:}
\end{equation}

对于奖励函数,我们也可以把动作去掉,这样就会得到类似于马尔可夫奖励过程的奖励函数,即
\begin{equation}
  r_{\pi}(s)=\sum_{a \in A} \pi(a \mid s) R(s, a)
  \label{eq:}
\end{equation}

\subsubsection{马尔可夫决策过程和马尔可夫过程/马尔可夫奖励过程的区别} 
马尔可夫决策过程里面的状态转移与马尔可夫奖励过程以及马尔可夫过程的状态转移的差异如\figref{fig:fig2.21} 所示。
马尔可夫过程/马尔可夫奖励过程的状态转移是直接决定的。比如当前状态是 $s$,那么直接通过转移概率决定下一个状态是什么。
但对于马尔可夫决策过程,它的中间多了一层动作 $a$ ,即智能体在当前状态的时候,首先要决定采取某一种动作,这样我们会到达某一个黑色的节点。到达这个黑色的节点后,因为有一定的不确定性,所以当智能体当前状态以及智能体当前采取的动作决定过后,智能体进入未来的状态其实也是一个概率分布。在当前状态与未来状态转移过程中多了一层决策性,这是马尔可夫决策过程与之前的马尔可夫过程/马尔可夫奖励过程很不同的一点。在马尔可夫决策过程中,动作是由智能体决定的,智能体会采取动作来决定未来的状态转移。

\begin{figure}[hbt]
  \centering
  \includegraphics[width=0.6\linewidth]{ch2/figs/2.21.png}
  \caption{马尔可夫决策过程与马尔可夫过程/马尔可夫奖励过程的状态转移的对比}
  \label{fig:fig2.21}
\end{figure}

    % \section{ DQN 算法}

我们知道求解强化学习的一种思路就是优化状态价值函数 $V(s)$ 或动作价值函数 $Q(s,a)$,而传统强化学习算法是以表格的形式存储这些价值函数的,这导致最终优化的对象本身就具有一定的局限性。首先表格形式的价值函数是离散的,存储的状态之间比较独立,对于一些近似状态的拟合就难以体现出关联性。以学生上下课为例,对于学生这个智能体来讲,数学课和物理课这两种状态其实应该是近似的,因为对于考高分这个长期目标来说,数学课和物理课具有类似的价值(都涉及复杂的公式和逻辑),换句话说对应的状态-价值函数应当类似。但是如果用表格形式存储的话,在更新价值函数的过程中是体现不出来这些关联性的。其次表格能够存储的状态-动作对是有限的,这受限于计算机的内存空间,这样一来在实际的强化学习任务中往往涉及成千上万个状态-动作对,如果都用表格来存储的话,会增加很多的计算和内存成本。在这种情况下,最好的办法就是换一种价值函数的表示形式,而深度神经网络就能很好地解决这个问题。深度神经网络本质上就是一个复杂的非线性函数,比较契合价值函数本身的定义,此外使用深度神经网络也能顺便解析一些复杂的状态及其内在的关联性,例如解决图像输入的卷积神经网络等等,因此目前主流的强化学习算法都是以深度神经网络为基础的算法,我们称作深度强化学习算法(DRL,Deep Reinforcement Learning)。

\subsection{深度学习基础}

在本章中会详细讲解一种经典的 DRL 算法,即 DQN 算法,在介绍该算法之前我们先简单了解一些深度学习的基础。注意,这里只是简单做一个入门的铺垫,如果读者们后面想深入做强化学习算法研究或者相关应用时,还是有必要进一步了解更多的深度网络模型的,例如能够识别图像状态输入的卷积神经网络、解析时间序列状态输入的循环神经网络等等。
\subsubsection{线性模型}

首先介绍线性模型,线性模型是最简单的一类机器学习模型,可以将其视为单层的神经网络。在线性模型中最基础的两个模型就是线性回归和逻辑回归,通常分别用于解决回归和分类问题,尽管后者也可以用来解决回归问题。回归模型的输出是一个连续的值,而分类模型的输出是离散的值,用于表示分类。本质上来说回归模型和分类模型是一样的,分类模型可以将回归模型离散化,例如本节将要讲的逻辑回归就是在线性回归的基础上增加了一个 sigmoid 函数对其进行了离散化。顺便提一句,回归模型也可以将分类模型连续化,通常见于贝叶斯模型中,但这在强化学习中并不常用。

{\bfseries 线性回归。} 以Kaggle入门竞赛项目房价预测为例,一套房子有$m$个特征,例如建造年份、房子面积等等,把这$m$个特征用向量表示,如下:

\begin{equation}
    % \label{eq:softmax_act}$
    \boldsymbol{x}=\left[x_1, x_2, \cdots, x_m\right]
\end{equation}

我们可以用线性模型来拟合这$m$个特征和房价的关系,如下:

\begin{equation}
    % \label{eq:softmax_act}$
    f(\boldsymbol{x} ; \boldsymbol{w}, b) = w_1 x_1+w_2 x_2+\cdots+w_m x_m+b = \boldsymbol{w}^T \boldsymbol{x}+b
\end{equation}

其中$\boldsymbol{w}$和$b$是模型的参数,$f(\boldsymbol{x} ; \boldsymbol{w}, b)$是模型的输出,也就是我们要预测的房价。出于简化考虑,通常我们会用一个符号$\boldsymbol{\theta}$来表示$\boldsymbol{w}$和$b$,如下:

\begin{equation}
    % \label{eq:softmax_act}$
    f^{\theta}(\boldsymbol{x}) = \boldsymbol{\theta}^T \boldsymbol{x}
\end{equation}

我们的目的是求得一组最优的参数$\boldsymbol{\theta^{*}}$,使得该模型能够根据房屋的$m$个特征预测对应的房价。我们一般利用历史数据来近似求解最优参数,这个过程就叫做训练。注意这里是近似求解,因为几乎所有机器学习模型都无法找到一种方法能够获得绝对的最优解,甚至也不一定存在绝对最优解,有些方法甚至很容易陷入局部最小值的问题中。训练的方法,或者说求解模型参数的方法理论上来说有很多种,比如这里线性模型可以用最小二乘法来求解,另外有些模型可以用牛顿法来求解,而目前普遍流行的优化方法就是梯度下降。梯度下降方法泛化能力很强,能够基于梯度下降求解很多种模型,该方法的本质是一阶泰勒展开,顺便提一句,牛顿法则是二阶泰勒展开。

{\bfseries 逻辑回归。} 对于分类问题,其预测目标不再是连续的值,而可能是二元变量,要么等于0,要么等于1,即最简单的二分类问题。这种情况下,我们可以用逻辑回归来解决,注意虽然逻辑回归名字里带有回归,但通常用于解决二分类问题而非回归问题。逻辑回归的思路也比较简单,如\figref{fig:logistic_struction}所示,就是在线性模型的后面增加一个sigmoid函数,我们一般称之为激活函数。逻辑回归模型其实可以看做神经网络模型的一个神经元,具体后面再展开说明。

\begin{figure}[hbt]
    \centering
    \includegraphics[width=0.5\linewidth]{ch4/figs/logistic_struction.png}
    \caption{逻辑回归,线性 Sigmoid 分类器的结构}
    \label{fig:logistic_struction}
\end{figure}

sigmoid 函数定义为:

\begin{equation}
    % \label{eq:softmax_act}$
    sigmoid(z) = \frac{1}{1+exp(-z)}
\end{equation}

如\figref{fig:sigmoid}所示,sigmoid 函数可以将输入的任意实数映射到$0-1$之间,对其输出的值进行判断,例如小于0.5我们认为预测的是类别0,反之是类别1,这样一来通过梯度下降来求解模型参数就可以用于实现二分类问题了。注意,虽然逻辑回归只是在线性回归模型基础上增加了一个激活函数,但两个模型是完全不同的,包括损失函数等等。线性回归的损失函数是均方差损失,而逻辑回归模型一般是交叉熵损失,这两种损失函数在深度学习和深度强化学习中都很常见,具体推导细节读者可自行翻阅相关资料。

\begin{figure}[hbt]
    \centering
    \includegraphics[width=0.5\linewidth]{ch4/figs/sigmoid.png}
    \caption{Sigmoid函数图像}
    \label{fig:sigmoid}
\end{figure}

\subsubsection{神经网络}

{\bfseries 全连接网络。} 全连接网络又称作多层感知机(multi-layer perceptron,MLP),是最基础的神经网络模型。它是基于生物神经网络的启发,将“线性函数+激活函数”这样的结构一层层堆叠(stack)组合成一个多层的网络模型,用于解决更复杂的问题。如\figref{fig:ann_vs_dnn}所示,线性函数可以看做生物神经网络的神经元,而激活函数就是神经元之间的突触结构。顺便说一句,生物神经网络是神奇且复杂的,人们也一直在尝试研究新的人工神经网络模型去模拟生物神经网络,例如脉冲神经网络,尽管目前这些模型还没有得到广泛地验证。

\begin{figure}[hbt]
    \centering
    \includegraphics[width=0.5\linewidth]{ch4/figs/ann_vs_dnn.png}
    \caption{生物神经网络与人工神经网络的对比}
    \label{fig:ann_vs_dnn}
\end{figure}

记神经网络模型中上一层的输入向量为$\boldsymbol{x^{l-1}}\in \mathbb{R}^{d^{l-1}}$,其中第一层的输入也就是整个模型的输入可记为$\boldsymbol{x^0}$,每一个全连接层将上一层的输入映射到$\boldsymbol{x^{l}}\in \mathbb{R}^{d^{l}}$,也就是下一层的输入,具体定义为:

\begin{equation}
    \boldsymbol{x}^{l}=\sigma(\boldsymbol{z}), \quad \boldsymbol{z}=\boldsymbol{W} \boldsymbol{x^{l-1}}+\boldsymbol{b} = \boldsymbol{\theta} \boldsymbol{x^{l-1}}
\end{equation}

其中$\boldsymbol{W}\in \mathbb{R}^{d^{l-1} \times d^{l}}$是权重矩阵,$\boldsymbol{b}$为偏置矩阵,与线性模型类型,这两个参数我们通常看作一个参数$\boldsymbol{\theta}$。$\sigma(\cdot)$是激活函数,除了 Sigmoid 函数之外,还包括 Softmax 函数、ReLU 函数和 tanh 函数等等激活函数。其中最常用的是 ReLU 函数 和 tanh 函数,前者将神经元也就是线性函数的输出映射到$0-1$之间,后者则映射到$-1$到$1$之间。前面讲到,在强化学习中我们用神经网络来近似动作价值函数,动作价值函数的输入是状态,输出是各个动作对应的价值,在有些连续动作问题中比如汽车方向盘转动角度是$-90$度到$90$度之间,这种情况下使用 tanh 激活函数能够使得神经网络负值以便于更好地近似状态动作函数。顺便提一句,这里还有一种做法是我们可以把动作空间映射到正值的范围,例如$0$到$180$区间,这样一来对应的神经网络模型激活函数使用 ReLU 函数会更好些。

一个$l$层的神经网络模型可以表示为:
\begin{equation}
    \begin{split}
    第 1 层: \quad \boldsymbol{x}^{(1)}=\sigma_1\left(\boldsymbol{W}^{(1)} \boldsymbol{x}^{(0)}+\boldsymbol{b}^{(1)}\right),\\
    第 2 层: \quad \boldsymbol{x}^{(2)}=\sigma_2\left(\boldsymbol{W}^{(2)} \boldsymbol{x}^{(1)}+\boldsymbol{b}^{(2)}\right),\\
    \vdots \quad \vdots\\
    第 l 层: \quad \boldsymbol{x}^{(l)}=\sigma_l\left(\boldsymbol{W}^{(l)} \boldsymbol{x}^{(l-1)}+\boldsymbol{b}^{(l)}\right)\\
\end{split}
\end{equation}

求解神经网络模型参数的方法除了梯度下降之外,还涉及多层网络模型的正向传播和反向传播,具体细节在接下来的小节中展开。

{\bfseries 卷积神经网络。}

{\bfseries 循环神经网络。}


{\bfseries Transformer。}
\subsubsection{梯度下降}

TODO
\subsubsection{反向传播}
TODO
\subsection{ DQN 算法}

DQN 算法,英文全称 Deep Q-learning,顾名思义就是基于深度网络模型的 Q-learning 算法,主要由 DeepMind 公司于2013年和2015年分别提出的两篇论文来实现,即《Playing Atari with Deep Reinforcement Learning》和《Human-level Control through Deep Reinforcement Learning》。DQN 算法相对于 Q-learning 算法来说更新方法本质上是一样的,而 DQN 算法最重要的贡献之一就是本章节开头讲的,用神经网络替换表格的形式来近似动作价值函数$Q(\boldsymbol{s},\boldsymbol{a})$。

\begin{figure}[hbt]
    \centering
    \includegraphics[width=0.5\linewidth]{ch4/figs/dqn_network.png}
    \caption{DQN 网络结构}
    \label{fig:dqn_network}
\end{figure}

如\figref{fig:dqn_network}所示,在 DQN 的网络模型中,我们将当前状态$s_t$作为输入,并输出动作空间中所有动作(假设这里只有两个动作,即1和2)对应的动作价值即$Q$值,我们记做$Q(s_t,\boldsymbol{a})$。对于其他状态,该网络模型同样可以输出所有动作对应的价值,这样一来神经网络近似的动作价值函数可以表示为$Q^{\theta}(\boldsymbol{s},\boldsymbol{a})$。其中$\theta$就是神经网络模型的参数,可以结合梯度下降的方法求解。

具体该怎么结合梯度下降来更新$Q$值呢?我们首先回顾一下 Q-learning 算法的更新公式如下:

\begin{equation}
    Q(s_t,a_t) \leftarrow Q(s_t,a_t)+\alpha[r_t+\gamma\max _{a}Q(s_{t+1},a)-Q(s_t,a_t)]
\end{equation}

我们注意到公式右边两项$r_t+\gamma\max _{a}Q(s_{t+1},a)$和$Q(s_t,a_t)$分别表示期望的$Q$值和实际的$Q$值,其中预测的$Q$值是用下一个状态对应$Q$值的最大值来近似的。换句话说,在更新$Q$值并达到收敛的过程中,期望的$Q$值也应该接近实际的$Q$值,即我们希望最小化$r_t+\gamma\max _{a}Q(s_{t+1},a)$和$Q(s_t,a_t)$之间的损失,其中$\alpha$是学习率,尽管优化参数的公式跟深度学习中梯度下降法优化参数的公式有一些区别(比如增加了$\gamma$和$r_t$等参数)。从这个角度上来看,强化学习跟深度学习的训练方式其实是一样的,不同的地方在于强化学习用于训练的样本(包括状态、动作和奖励等等)是与环境实时交互得到的,而深度学习则是事先准备好的训练集。当然训练方式类似并不代表强化学习和深度学习之间的区别就很小,本质上来说强化学习和深度学习解决的问题是完全不同的,前者用于解决序列决策问题,后者用于解决静态问题例如回归、分类、识别等等。在 Q-learning 中,我们是直接优化 Q 值的,而在 DQN 中使用神经网络来近似 Q 值,我们则需要优化网络模型对应的参数$\theta$,如下

\begin{equation}
    \begin{split}
    y_{i}= \begin{cases}r_{i} & \text {对于终止状态} s_{i} \\ r_{i}+\gamma \max _{a^{\prime}} Q\left(s_{i+1}, a^{\prime} ; \theta\right) & \text {对于非终止状态} s_{i}\end{cases}\\
    L(\theta)=\left(y_{i}-Q\left(s_{i}, a_{i} ; \theta\right)\right)^{2}\\
    \theta_i \leftarrow \theta_i - \alpha \nabla_{\theta_{i}} L_{i}\left(\theta_{i}\right)\\
\end{split}
\end{equation}

其中期望的Q值$y_{i}$增加了对终止状态和非终止状态的判断,这是因为当$s_t$为终止状态时,$Q(s_{t+1},a)$是不存在的($s_{t+1}$不存在),所以需要将其置0,在 Q-learning 中其实也需要有同样的操作,只是出于简化考虑没有列出。

\subsubsection{经验回放}

前面讲到,强化学习的训练方式是与环境实时交互得到样本然后进行训练的,在 Q-learning 中我们是每次交互一个样本,通常包含当前状态($state$)、当前动作($action$)、下一个状态($next\_state$)、是否为终止状态($done$),这样一个样本我们一般称之为一个状态转移(transition)。这种每次只交互一个样本并更新的方式会产生两个问题,首先是在强化学习问题中样本之间的关联性过强(从当前状态到下一个状态的过渡不可能是突变的,比如我们在吃饭时中间忽然加快速度张开血盆大口猛吃也不会导致我们的肚子一下子就鼓鼓的,会有一个相对缓慢的变化过程),会导致更新的过程不够稳定并且容易陷入局部最优解。本质上来说这其实就是随机梯度下降相对于单纯梯度下降的好处,只是在强化学习中体现得更为明显,因为强化学习前后两个样本的关联性往往比监督学习更紧密,从而导致训练的不稳定。另外一个问题是我们每次只看一个样本来更新,这在 DQN 算法中的劣势会更加明显,因为 DQN 算法是基于深度神经网络模型的。在深度学习的梯度下降中,我们知道如果在训练时每次遍历整个数据集并更新一次损失函数和梯度是会保证不错的收敛性的,但是计算开销会很大,这就是前面所说的批梯度下降方法(batch gradient descent)。但如果每次只看一个样本并更新梯度,尽管速度会提上去,但是会导致收敛性能不好,容易在最优点附近徘徊,于是有了一个折中的方法,即小批量梯度下降(mini-batch gradient descent)。鉴于这两个问题, DeepMind 公司 在论文中提出了一个经验回放的概念(replay buffer),这个经验回放的功能主要包括三个方面。首先是能够缓存一定量的状态转移即样本,此时 DQN 算法并不急着更新并累积一定的初始样本,就好比我们学习做饭炒菜一样,先把颠勺、抡刀、切菜、放调料等等都先零碎地试一遍。然后是每次更新的时候随机从经验回放中取出一个小批量的样本并更新策略,注意这里的随机和小批量以便保证我们存储动作价值函数的网络模型是小批量随机梯度下降的。最后要保证经验回放是具有一定的容量限制的,太小了会导致收集到的样本具有一定的局限性,太大了会失去经验本身的意义。这就好比我们在做自我规划一样,往往会根据自身经验制定一个三年或者五年计划,如果制定的计划周期太长比如制定一个二十年计划是没有任何意义的,因为二十年间的变数太多会导致制定的计划失去效用。类似的经验回放太大容易导致智能体在更新策略时可能会使用一些比较久远的样本,根据马尔可夫过程的性质,太过久远的样本对于当前状态的参考意义不大,就好比我们制定一个二十年当上小学校长的计划,等到十年过去后发现我们中间有太多的变数而身不由己,这样一来当初制定的二十年计划大概率就流产了。而小批量的样本更新就好比在三五年计划的基础上我们制定一个每周计划,每周行动并反思然后结合三五年计划或目标调整下一周的行动,这种方式往往是最高效的。到这里,我们又不得不感叹一句,生活中处处是强化学习!
\subsubsection{目标网络}

在 DQN 算法中还有一个重要的技巧,就是使用了一个每隔若干步才更新的目标网络,与之相对的,会有一个每步更新的网络,即每次从经验回放中采样到样本就更新网络参数,在本书中一般称之为策略网络。策略网络和目标网络结构都是相同的,都用于近似 Q 值,在实践中每隔若干步才把每步更新的策略网络参数复制给目标网络,这样做的好处是保证训练的稳定,避免 Q值 的估计发散。举一个典型的例子,这里的目标网络好比明朝的皇帝,而策略网络相当于皇帝手下的太监,每次皇帝在做一些行政决策时往往不急着下定论,会让太监们去收集一圈情报,然后集思广益再做决策。这样做的好处是显而易见的,比如皇帝要处决一个可能受冤的犯人时,如果一个太监收集到一个情报说这个犯人就是真凶的时候,如果皇帝是一个急性子可能就当初处决了,但如果这时候另外一个太监收集了一个更有力的证据证明刚才那个太监收集到的情报不可靠并能够证明该犯人无罪时,那么此时皇帝就已经犯下了一个无法挽回的过错。换句话说,如果当前有个小批量样本导致模型对 Q 值进行了较差的过估计,如果接下来从经验回放中提取到的样本正好连续几个都这样的,很有可能导致 Q 值的发散(它的青春小鸟一去不回来了)。再打个比方,我们玩 RPG 或者闯关类游戏,有些人为了破纪录经常存档(Save)和回档(Load),简称“SL”大法。只要我出了错,我不满意我就加载之前的存档,假设不允许加载呢,就像 DQN 算法一样训练过程中会退不了,这时候是不是搞两个档,一个档每帧都存一下,另外一个档打了不错的结果再存,也就是若干个间隔再存一下,到最后用间隔若干步数再存的档一般都比每帧都存的档好些呢。当然我们也可以再搞更多个档,也就是DQN增加多个目标网络,但是对于 DQN 算法来说没有多大必要,因为多几个网络效果不见得会好很多。


\subsubsection{探索策略}
\subsection{实战:DQN算法}


\subsection{关键词}
\subsection{习题}
\subsection{面试题}
\subsection{本章小结}
    % \section{策略梯度}
本章开始介绍基于策略梯度(policy based)的算法,与前面介绍的基于价值(value based)的算法(包括Q learning,SARSA以及DQN等等)不同,这类算法直接对策略本身进行近似优化。在这种情况下,我们可以将策略描述成一个带有参数$\theta$的连续函数,该函数将某个状态作为输入,输出的不再是某个确定性(deterministic)的离散动作,而是对应的动作概率分布,通常用$\pi_{\theta}(a|s)$表示,称作随机性(stochastic)策略。下面我们将从最基本的策略梯度算法展开。

\subsection{策略梯度算法}

尽管策略梯度算法是将策略参数化成一个连续的函数$\pi_{\theta}(a|s)$,但是与基于价值的算法本质上是一样的,最终的优化目标都是累积的价值期望$V^{*}(s)$。例如在前面章节的 Q learning 算法中,我们利用贝尔曼方程求解马尔科夫决策过程中的最佳决策序列,进而求出对应的最优动作价值函数$Q^{*}(s,a)$,不清楚的读者可以再回到前面的内容温习一下。

策略一般记作 $\pi$。假设我们使用深度学习来做强化学习,策略就是一个网络。网络里面有一些参数,我们用 $\theta$ 来代表 $\pi$ 的参数。
网络的输入是智能体看到的东西,如果让智能体玩视频游戏,智能体看到的东西就是游戏的画面。智能体看到的东西会影响我们训练的效果。例如,在玩游戏的时候, 也许我们觉得游戏的画面是前后相关的,所以应该让策略去看从游戏开始到当前这个时间点之间所有画面的总和。因此我们可能会觉得要用到循环神经网络(recurrent neural network,RNN)来处理它,不过这样会比较难处理。
我们可以用向量或矩阵来表示智能体的观测,并将观测输入策略网络,策略网络就会输出智能体要采取的动作。
\figref{fig:actor_policy} 就是具体的例子,策略是一个网络;输入是游戏的画面,它通常是由像素组成的;输出是我们可以执行的动作,有几个动作,输出层就有几个神经元。假设我们现在可以执行的动作有 3 个,输出层就有 3 个神经元,每个神经元对应一个可以采取的动作。输入一个东西后,网络会给每一个可以采取的动作一个分数。我们可以把这个分数当作概率,演员根据概率的分布来决定它要采取的动作,比如 0.7 的概率向左走、0.2 的概率向右走、0.1的概率开火等。概率分布不同,演员采取的动作就会不一样。

\begin{figure}[hbt]
    \centering
    \includegraphics[width=0.5\linewidth]{ch6/figs/actor_policy.png}
    \caption{演员的策略}
    \label{fig:actor_policy}
\end{figure}

接下来我们用一个例子来说明演员与环境交互的过程。

如\figref{fig:example_play_game}所示,首先演员会看到一个视频游戏的初始画面,接下来它会根据内部的网络(内部的策略)来决定一个动作。假设演员现在决定的动作是向右,决定完动作以后,它就会得到一个奖励,奖励代表它采取这个动作以后得到的分数。

我们把游戏初始的画面记作 $s_0$, 把第一次执行的动作记作 $a_0$,把第一次执行动作以后得到的奖励记作 $r_1$。
这里不同的人有不同的记法,有人觉得应该从 $s_1$ 开始,并且执行 $a_1$ 得到的奖励应该记为 $r_1$,这两种记法都可以。只是一般来说,是智能体先做出动作,然后环境再输出奖励和新的状态,也就是说这个新的状态和奖励应该对应着同一时刻,也就是同一下标。我们通常喜欢从 0 时刻开始记录轨迹,此轨迹可以将交互过程清晰地体现出来。
演员决定一个动作以后,就会看到一个新的游戏画面$s_1$。把 $s_1$ 输入给演员,演员决定要开火,它可能打败了一只怪兽,就得到五分。这个过程反复地持续下去,直到在某一个时间点执行某一个动作,得到奖励之后,环境决定这个游戏结束。例如,如果在这个游戏里面,我们控制宇宙飞船去击杀怪兽,如果宇宙飞船被毁或是把所有的怪兽都清空,游戏就结束了。

\begin{figure}[hbt]
    \centering
    \includegraphics[width=0.5\linewidth]{ch6/figs/example_play_game.png}
    \caption{玩视频游戏的例子}
    \label{fig:example_play_game}
\end{figure}


那么在策略梯度算法中我们怎样去推导出最优策略下的价值期望呢?首先我们知道强化学习解决的问题基本上都可以被描述为马尔可夫决策过程,而马尔可夫决策过程同时也是成环境与智能体不断交互的过程,即环境与策略不断交互的过程,因为智能体是策略的载体。如\figref{fig:env_agent} 所示,环境首先“吐”出一个初始状态$s_0$,然后智能体观测到状态$s_0$,它会“吐”出相应的动作 $a_0$。紧接着环境接收到 $a_0$ 反馈出新的状态 $s_1$,然后智能体观测到新的状态继续采取新的动作......如此循环往复,直到满足环境的停止条件为止。

\begin{figure}[hbt]
    \centering
    \includegraphics[width=0.5\linewidth]{ch6/figs/env_agent.png}
    \caption{智能体与环境}
    \label{fig:env_agent}
\end{figure}

在这样一个过程之后,我们把环境输出的所有状态 $s$ 与智能体输出的对应动作 $a$ 全部组合起来,就形成了一条轨迹(trajectory),即
\begin{equation}
    \label{eq:}
    \tau=\left\{s_{0}, a_{0}, s_{1}, a_{1}, \cdots, s_{t}, a_{t}\right\}
\end{equation}

这里环境也可以看作一个函数,我们设在给定状态$s_t$和动作$a_t$的情况下,它“吐”出状态$s_{t+1}$的概率为$p(s_{t+1} | s_{t}, a_{t})$,即是马尔可夫决策过程中的转移概率。此外我们假设环境以一个概率$p(s_{0})$“吐”出初始状态,在给定策略函数$\pi_{\theta}(a|s)$的情况下,我们就可以计算某个轨迹$\tau$发生的概率为
\begin{equation}
    \label{eq:station_dist}
    \begin{aligned}
        P_{\theta}(\tau)
        &=p(s_{0}) \pi_{\theta}(a_{0} | s_{0}) p(s_{1} | s_{0}, a_{0}) \pi_{\theta}(a_{1} | s_{1}) p(s_{2} | s_{1}, a_{1}) \cdots \\
        &=p(s_{0}) \prod_{t=0}^{T} \pi_{\theta}\left(a_{t} | s_{t}\right) p\left(s_{t+1} | s_{t}, a_{t}\right)
    \end{aligned}
\end{equation}

也就是说我们先计算环境输出初始状态 $s_0$ 的概率 $p(s_0)$,再计算智能体根据 $s_0$ 动作执行 $a_0$ 的概率,也就是策略函数$p_{\theta}\left(a_{0} | s_{0}\right)$。然后环境结合当前状态$s_0$根据动作$a_0$以一定的概率反馈出新的状态 $s_1$,也就是转移概率$p(s_{1} | s_{0}, a_{0})$。 
这样的转移概率通常情况下是一定存在的,也就是不为0,因为 $s_1$ 与 $s_0$ 一般说来是有关系的。举个例子,我们玩电脑游戏时,一般都是根据每一帧的游戏画面给出的信息来采取我们认为的最优动作以便于获得游戏的最终胜利。这种情况下,游戏画面可以看作状态,比如当前帧画面为$s_0$,下一帧画面就是 $s_1$ 。由于游戏画面是连续的,因此下一帧游戏画面$s_1$ 与上一帧游戏画面$s_0$ 通常是有联系的。如果显示器输出游戏画面的时候没有概率,极端情况下游戏的画面就会始终停留在某一帧下,此时我们只要找到一条路径就可以过关了,这样的游戏就没有意义。所以输出游戏画面时通常有一定概率,给定同样的前一个画面,我们采取同样的动作,下次产生的画面不一定是一样的。如此反复执行下去,我们就可以计算出一条轨迹 $\tau$ 出现的概率了。

前面讲到,无论是基于价值还是基于策略梯度的方法,我们的目标都是希望最终累积的价值期望最大,这个价值的定义或者说近似也是比较多样的,可以是简单的累积奖励之和,也可以是包含折扣因子$\gamma$的累积奖励之和,也就是通常我们所说的回报$G$(return)。如何近似这个价值期望也是研究者们近年来一直在不断地优化策略梯度算法的重点之一,这个在后面章节我们讲A2C和GAE等算法的时候会继续展开,现在我们姑且将这个价值近似为最为简单的累积奖励。

再简单回顾一下马尔可夫决策过程。我们知道强化学习中除了智能体和环境之外,还会涉及奖励。奖励一般是由环境反馈得到的,如\figref{fig:expected_reward} 所示,环境根据在某个状态以及在这一状态下智能体采取的某个动作来决定这个动作可以得到的分数,也就是奖励。例如,输入$s_0$、$a_0$,它会输出$r_1$;输入 $s_1$、$a_1$,会输出 $r_2$ 等等,如\eqref{eq:traj_reward}。

\begin{equation}
    \label{eq:traj_reward}
    \tau=\left\{s_{0}, a_{0}, s_{1},r_{1},a_{1},\cdots, s_{t-1}, r_{t-1}, a_{t-1},s_{t},r_{t}\right\}
\end{equation}

\begin{figure}[hbt]
    \centering
    \includegraphics[width=0.5\linewidth]{ch6/figs/expected_reward.png}
    \caption{期望的奖励}
    \label{fig:expected_reward}
\end{figure}

奖励一般也可以近似成关于状态和动作的函数,即$r_{t+1}=r(s_t,a_t),t=0,1,\cdots$。对于一条轨迹$\tau$,我们可以计算其对应的累积奖励为$R(\tau)=\sum_{t=0}^T r\left(s_t, a_t\right)$。那么在给定的策略下,即参数$\theta$固定,对于不同的初始状态,会形成不同的轨迹$\tau_{1},\tau_{2},\cdots$,对应轨迹的出现概率前面已经推导出来为$P_{\theta}(\tau_{1}),P_{\theta}(\tau_{2}),\cdots$,累积奖励则为$R(\tau_{1}),R(\tau_{2}),\cdots$。回忆一下概率论中的全期望公式,是不是该策略的价值期望公式就可以通过每条轨迹的概率乘上对应的累积奖励再求和得到呢?答案是肯定的!如\eqref{eq:expect_policy},这就是我们所要找的目标函数。

\begin{equation}
    \label{eq:expect_policy}
    \begin{aligned}
    J(\pi_{\theta}) = P_{\theta}(\tau_{1})R(\tau_{1})+P_{\theta}(\tau_{2})R(\tau_{2})+\cdots \\
    &=\int_\tau P_{\theta}(\tau) R(\tau) \\ 
    &=E_{\tau \sim P_\theta(\tau)}[\sum_t r(s_t, a_t)] \\
    &=\underset{\tau \sim \pi_\theta}{E}[R(\tau)] 
    \end{aligned}
\end{equation}

有了目标函数,读者如果学过深度学习就会很自然地想到用梯度下降或者上升的方法来求解对应的最优参数$\theta^{*}$,这里需要用到梯度上升法,因为我们的目标是让总的累积价值期望$J(\pi_{\theta})$最大,而不是最小。当然我们也可以将目标函数取负号,即求解$-J(\pi_{\theta})$,然后再用梯度下降法去求解。梯度下降是更为普遍的做法,熟悉Tensorflow或者PyTorch等框架的读者应该比较清楚,这些框架默认的优化器设置就是梯度下降的,因此实际编程的时候如果我们的目标是最大化某个量,我们就会取这个量的相反数来优化。再比如用过scipy模块中的linprog函数来求解线性规划的读者也会知道,linprog函数默认的设置是求目标函数的最小值,而不是最大值,当我们的实际问题是求最大值时,我们也会进行取反的操作。

回归正题,梯度下降方法的关键还是在于求出$J(\pi_{\theta})=\int_\tau P_{\theta}(\tau) R(\tau)$的梯度,但是一眼看上去似乎不太好求。我们先不着急,首先我们是求关于参数$\theta$的梯度,可以看到$R(\tau)$跟$\theta$其实是没有关联的,因此在求解梯度的时候可以将这一项看作常数。那么接下来就是怎么求$P_{\theta}(\tau)$关于$\theta$的梯度了,我们需要用到一个对数微分的技巧,即$\log x$的导数是$1/x$。注意这里我们通常默认$\log$的底数是$e$,因此这里$\log x$也就是我们常见的$\ln x$,在大学数学之后的概念中我们通常是写作$\log x$,同学们请务必习惯。根据这个技巧,我们就可以推出\eqref{eq:log_trick}。
\begin{equation}
    \label{eq:log_trick}
    \nabla_\theta P_{\theta}(\tau)= P_{\theta}(\tau) \frac{\nabla_\theta P_{\theta}(\tau)}{P_{\theta}(\tau) }= P_{\theta}(\tau) \nabla_\theta \log P_{\theta}(\tau)
\end{equation}

现在的问题就从求$P_{\theta}(\tau)$的梯度变成了求$\log P_{\theta}(\tau)$的梯度了,即求$\nabla_\theta \log P_{\theta}(\tau)$。我们先求出$\log P_{\theta}(\tau)$,根据\eqref{eq:station_dist},$P_{\theta}(\tau)=p(s_{0}) \prod_{t=0}^{T} \pi_{\theta}\left(a_{t} | s_{t}\right) p\left(s_{t+1}  s_{t}, a_{t}\right)$,再根据对数公式$log (ab) = log a + log b$,即可求出:

\begin{equation}
    \label{eq:station_dist_log}
    \log P_{\theta}(\tau)= \log p(s_{0})  +  \sum_{t=0}^T(\log \pi_{\theta}(a_t \mid s_t)+\log p(s_{t+1} \mid s_t,a_t))
\end{equation}

我们惊奇地发现$\log P_{\theta}(\tau)$展开之后只有中间的项$\log \pi_{\theta}(a_t \mid s_t)$跟参数$\theta$有关,也就是说其他项关于$\theta$的梯度为0,如\eqref{eq:station_dist_log_grad}所示。
\begin{equation}
    \label{eq:station_dist_log_grad}
    \begin{aligned}
    \nabla_\theta \log P_{\theta}(\tau) &=\nabla_\theta \log \rho_0\left(s_0\right)+\sum_{t=0}^T\left(\nabla_\theta \log \pi_\theta\left(a_t \mid s_t\right)+\nabla_\theta \log p\left(s_{t+1} \mid s_t, a_t\right)\right) \\
    &=0+\sum_{t=0}^T\left(\nabla_\theta \log \pi_\theta\left(a_t \mid s_t\right)+0\right) \\
    &=\sum_{t=0}^T \nabla_\theta \log \pi_\theta\left(a_t \mid s_t\right)
    \end{aligned}
\end{equation}

现在我们就可以很方便地求出目标函数的梯度了,如\eqref{eq:pg_ob_grad}所示。

\begin{equation}
    \label{eq:pg_ob_grad}
    \begin{aligned}
    \nabla_\theta J\left(\pi_\theta\right) &=\nabla_\theta \underset{\tau \sim \pi_\theta}{\mathrm{E}}[R(\tau)] \\
    &=\nabla_\theta \int_\tau P_{\theta}(\tau) R(\tau) \\
    &=\int_\tau \nabla_\theta P_{\theta}(\tau) R(\tau) \\
    &=\int_\tau P_{\theta}(\tau) \nabla_\theta \log P_{\theta}(\tau) R(\tau) \\
    &=\underset{\tau \sim \pi_\theta}{\mathrm{E}}\left[\nabla_\theta \log P_{\theta}(\tau) R(\tau)\right]\\
    &= \underset{\tau \sim \pi_\theta}{\mathrm{E}}\left[\sum_{t=0}^T \nabla_\theta \log \pi_\theta\left(a_t \mid s_t\right) R(\tau)\right]
    \end{aligned}
\end{equation}

我们再简单解释一下\eqref{eq:pg_ob_grad}中的步骤,首先第一行就是目标函数的表达形式,到第二行就是全期望展开式,到第三行就是利用了积分的梯度性质,即梯度可以放到积分号的里面也就是被积函数中,第四行到最后就是对数微分技巧了。回过头来看下,我们为什么要用到对数微分技巧呢?这其实是一个常见的数学技巧:当我们看到公式中出现累乘的项时,我们通常都会取对数简化,因为根据对数公式的性质可以将累乘的项转换成累加的项,这样一来问题会更加便于处理。

我们可以直观地理解\eqref{eq:pg_ob_grad},即在采样到的数据里面,$(s_t,a_t)$ 可看作整个轨迹 $\tau$ 里面的某一个状态$-$动作对,假设我们在 $s_t$ 下执行 $a_t$时,最后发现 $\tau$ 的奖励是正的,我们就要增加在 $s_t$ 下执行 $a_t$ 的概率。反之,如果若 $\tau$ 的奖励是负的, 我们就要减少在 $s_t$ 下执行 $a_t$ 的概率,这其实就是梯度的思想。前面讲到我们将目标函数取反,就可以用梯度下降方法求取最优的参数$\theta$,即
\begin{equation}
    \label{eq:theta_update}
    \theta \leftarrow \theta - \alpha (-\nabla_\theta J\left(\pi_\theta\right))
\end{equation}
其中$\alpha$表示学习率,不明白\eqref{eq:theta_update}的读者就可以再去温习一下梯度下降法的原理,实际编程过程中我们还可以用 Adam、RMSProp 等方法来调整学习率,具体见后面的实战内容。

\subsection{REINFORCE算法}

在本小节中我们将介绍策略梯度中最简单也是最经典的一个算法\kw{REINFORCE},又称蒙特卡洛策略梯度算法(Monte Carlo Policy Gradient)。在介绍该算法之前,我们先回顾一下蒙特卡洛方法。

如\figref{fig:mc_td} 所示,蒙特卡洛方法可以理解为算法完成一个回合之后,再利用这个回合的数据去更新策略,也就是学习。因为我们已经获得了整个回合的数据,相应地也能够知道每一个步骤的奖励,我们可以很方便地计算每个步骤的未来总奖励,即回报 $G_t$。$G_t$ 是未来总奖励,代表从这个步骤开始,我们能获得的奖励之和。$G_1 $代表我们从第一步开始,往后能够获得的总奖励。$G_2$ 代表从第二步开始,往后能够获得的总奖励。

相比蒙特卡洛方法一个回合更新一次,时序差分方法是每个步骤更新一次,即每走一步,更新一次,时序差分方法的更新频率更高。时序差分方法使用Q函数来近似地表示未来总奖励 $G_t$。
\begin{figure}[hbt]
    \centering
    \includegraphics[width=0.5\linewidth]{ch6/figs/mc_td.png}
    \caption{蒙特卡洛方法与时序差分方法}
    \label{fig:mc_td}
\end{figure}

REINFORCE 算法正是利用蒙特卡洛算法来计算每回合生成的轨迹的价值,即$G_t$,然后乘上对应轨迹的概率从而计算总的价值期望,即\eqref{eq:reinforce_update}。

\begin{equation}
    \label{eq:reinforce_update}
    \nabla J_{\theta} \approx \frac{1}{N} \sum_{n=1}^{N} \sum_{t=1}^{T_{n}} G_{t}^{n} \nabla \log \pi_{\theta}\left(a_{t}^{n} \mid s_{t}^{n}\right)
\end{equation}

我们回顾一下前面策略梯度基础推导中以轨迹为对象的公式,即\eqref{eq:pg_ob_grad_2},如下:

\begin{equation}
    \begin{aligned}
    \nabla_\theta J\left(\pi_\theta\right) = \underset{\tau \sim \pi_\theta}{\mathrm{E}}\left[\sum_{t=0}^T \nabla_\theta \log \pi_\theta\left(a_t \mid s_t\right) R(\tau)\right]
    \end{aligned}
\end{equation}

其实这跟 REINFORCE 算法的目标函数梯度公式 (\eqref{eq:reinforce_update})几乎是等效的,换句话说,REINFORCE 算法本质上就是计算轨迹的概率和对应轨迹的累积价值然后得到总的价值期望,是一种比较纯的原始的策略梯度算法。前面我们讲到计算轨迹对应的奖励是十分繁琐的,因为我们不确定要计算多少步。但是 REINFORCE 算法这里与前面策略梯度基础推导公式不同的是,它使用了带有折扣因子$\gamma$的回报$G_t$来代替单纯的累积奖励$R(\tau)$去近似对应轨迹的累积奖励。在之前关于贝尔曼公式的推导中,我们其实已经领教过了这种带折扣因子的魅力。这种折扣因子可以让当前时刻和下一时刻的价值很好地联系起来,从而避免无限循环状态下的计算问题,即$V_t = r_{r+1}+\gamma V_{t+1}$,从而能够很方便地进行后面的推导。这里的回报$G_t$同理,如\eqref{eq:future_relation}。
\begin{equation}
    \label{eq:future_relation}
    \begin{aligned}
        G_{t} &=\sum_{k=t+1}^{T} \gamma^{k-t-1} r_{k} \\
        &=r_{t+1}+\gamma G_{t+1}
        \end{aligned}
\end{equation}

建立起了这样一种联系之后,在数学分析或编写代码上,我们是从后往前推,即从$G_T$一步一步地往前推到$G_1$。相比于原来每个轨迹都要从头开始计算对应的$R(\tau)$,已经在简化的道路上迈出了不小的一步了。

梯度公式弄清楚之后,我们就可以直接写出对应的伪代码了。如\figref{fig:REINFORCE} 所示,
REINFORCE 的伪代码主要看最后4行,先产生一个回合的数据,比如 
$$
(s_1,a_1,G_1),(s_2,a_2,G_2),\cdots,(s_T,a_T,G_T)
$$
然后针对每个动作计算梯度 $\nabla \ln \pi(a_t|s_t,\theta)$ 。在代码中,我们要获取神经网络的输出。神经网络会输出每个动作对应的概率值(比如0.2、0.5、0.3),然后我们还可以获取实际的动作$a_t$,并转成独热(one-hot)向量(比如[0,1,0])与 $\log ([0.2,0.5,0.3])$ 相乘就可以得到 $\ln \pi(a_t|s_t,\theta)$  。
\begin{figure}[hbt]
    \centering
    \includegraphics[width=0.5\linewidth]{ch6/figs/REINFORCE.png}
    \caption{REINFORCE算法}
    \label{fig:REINFORCE}
\end{figure}

\begin{tcolorbox}[colframe=blue!25,colback=blue!10]
独热编码(one-hot encoding)通常用于处理类别间不具有大小关系的特征。 例如血型,一共有4个取值(A型、B型、AB型、O型),独热编码会把血型变成一个4维稀疏向量,A型血表示为$[1,0,0,0]$,B型血表示为$[0,1,0,0]$,AB型血表示为$[0,0,1,0]$,O型血表示为$[0,0,0,1]$,\upcite{zhugesheng}。
\end{tcolorbox}

如\figref{fig:mnist_recognition} 所示,
手写数字识别是一个经典的多分类问题,输入是一张手写数字的图片,经过神经网络处理后,输出的是各个类别的概率。我们希望输出的概率分布尽可能地贴近真实值的概率分布。
因为真实值只有一个数字 9,所以如果我们用独热向量的形式给它编码,也可以把真实值理解为一个概率分布,9 的概率就是1,其他数字的概率就是 0。
神经网络的输出一开始可能会比较平均,通过不断地迭代、训练优化之后,我们会希望输出9 的概率可以远高于输出其他数字的概率。

\begin{figure}[hbt]
    \centering
    \includegraphics[width=0.5\linewidth]{ch6/figs/mnist_recognition.png}
    \caption{监督学习例子:手写数字识别}
    \label{fig:mnist_recognition}
\end{figure}

如\figref{fig:improve_nine_prob} 所示,我们所要做的就是提高输出 9 的概率,降低输出其他数字的概率,让神经网络输出的概率分布能够更贴近真实值的概率分布。我们可以用交叉熵来表示两个概率分布之间的差距。

\begin{figure}[hbt]
    \centering
    \includegraphics[width=0.5\linewidth]{ch6/figs/improve_nine_prob.png}
    \caption{提高数字9的概率}
    \label{fig:improve_nine_prob}
\end{figure}

我们看一下监督学习的优化流程,即怎么让输出逼近真实值。
如\figref{fig:opti_process} 所示,
监督学习的优化流程就是将图片作为输入传给神经网络,神经网络会判断图片中的数字属于哪一类数字,输出所有数字可能的概率,再计算交叉熵,即神经网络的输出 $Y_i$ 和真实的标签值 $Y_i'$ 之间的距离 $-\sum Y_{i}^{\prime} \cdot \log \left(Y_{i}\right)$。我们希望尽可能地缩小这两个概率分布之间的差距,计算出的交叉熵可以作为损失函数传给神经网络里面的优化器进行优化,以自动进行神经网络的参数更新。
\begin{figure}[hbt]
    \centering
    \includegraphics[width=0.5\linewidth]{ch6/figs/opti_process.png}
    \caption{优化流程}
    \label{fig:opti_process}
\end{figure}

类似地,如\figref{fig:pg_loss} 所示,策略梯度预测每一个状态下应该要输出的动作的概率,即输入状态 $s_t$,输出动作$a_t$的概率,比如 0.02、0.08、0.9。实际上输出给环境的动作是随机选择一个动作,比如我们选择向右这个动作,它的独热向量就是(0,0,1)。
我们把神经网络的输出和实际动作代入交叉熵的公式就可以求出输出动作的概率和实际动作的概率之间的差距。
但实际的动作 $a_t$ 只是我们输出的真实的动作,它不一定是正确的动作,它不能像手写数字识别一样作为一个正确的标签来指导神经网络朝着正确的方向更新,所以我们需要乘一个奖励回报 $G_t$。$G_t$相当于对真实动作的评价。
如果 $G_t$ 越大,未来总奖励越大,那就说明当前输出的真实的动作就越好,损失就越需要重视。
如果 $G_t$ 越小,那就说明动作 $a_t$ 不是很好,损失的权重就要小一点儿,优化力度也要小一点儿。
通过与手写数字识别的一个对比,我们就知道为什么策略梯度损失会构造成这样。
\begin{figure}[hbt]
    \centering
    \includegraphics[width=0.5\linewidth]{ch6/figs/pg_loss.png}
    \caption{策略梯度损失}
    \label{fig:pg_loss}
\end{figure}

如\figref{fig:loss_compute} 所示,
实际上我们在计算策略梯度损失的时候,要先对实际执行的动作取独热向量,
% 拿到那个 $\ln \pi(a_t|s_t,\theta)$。我就拿实际执行的这个动作,
再获取神经网络预测的动作概率,将它们相乘,我们就可以得到 $\ln \pi(a_t|s_t,\theta)$,这就是我们要构造的损失。
因为我们可以获取整个回合的所有的轨迹,所以我们可以对这一条轨迹里面的每个动作都去计算一个损失。把所有的损失加起来,我们再将其“扔”给 Adam 的优化器去自动更新参数就好了。
\begin{figure}[hbt]
    \centering
    \includegraphics[width=0.5\linewidth]{ch6/figs/loss_compute.png}
    \caption{损失计算}
    \label{fig:loss_compute}
\end{figure}

\figref{fig:REINFORCE_process} 所示为 REINFORCE 算法示意图,首先我们需要一个策略模型来输出动作概率,输出动作概率后,
通过 \kw{sample()} 函数得到一个具体的动作,与环境交互后,我们可以得到整个回合的数据。得到回合数据之后,我们再去执行\kw{learn()} 函数,在 \kw{learn()} 函数里面,我们就可以用这些数据去构造损失函数,“扔”给优化器优化,更新我们的策略模型。
\begin{figure}[hbt]
    \centering
    \includegraphics[width=0.5\linewidth]{ch6/figs/REINFORCE_process.png}
    \caption{REINFORCE算法示意}
    \label{fig:REINFORCE_process}
\end{figure}

\subsection{策略梯度的进阶推导}

在上一节中我们通过计算轨迹的概率并乘上对应的价值,然后将通过全期望公式将这些项累加起来就得到了关于策略的总价值期望,即我们要求得的目标函数,如\eqref{eq:expect_policy_2}所示。
\begin{equation}
    \label{eq:expect_policy_2}
    \begin{aligned}
    J(\pi_{\theta}) = P_{\theta}(\tau_{1})R(\tau_{1})+P_{\theta}(\tau_{2})R(\tau_{2})+\cdots \\
    &=\int_\tau P_{\theta}(\tau) R(\tau) \\ 
\end{aligned}
\end{equation}
然后通过对数微分等技巧求出对应的梯度,如\eqref{eq:pg_ob_grad_2}所示。
\begin{equation}
    \label{eq:pg_ob_grad_2}
    \begin{aligned}
    \nabla_\theta J\left(\pi_\theta\right) = \underset{\tau \sim \pi_\theta}{\mathrm{E}}\left[\sum_{t=0}^T \nabla_\theta \log \pi_\theta\left(a_t \mid s_t\right) R(\tau)\right]
    \end{aligned}
\end{equation}

那么问题来了,可以看到无论是目标函数还是对应的梯度都必须求出关于轨迹$\tau$的项,例如$R(\tau)$。乍一看这个包含轨迹的项似乎比较容易求,但实际上是很难的,就像我们某些玩家日常指挥别人打游戏时那样,嘴上说说容易,实际操作起来却发现处处是细节。为什么说处处是细节呢?注意看\eqref{eq:pg_ob_grad_2}中对于$\log \pi_\theta\left(a_t \mid s_t\right)$中的每一个状态-动作对$(s_t,a_t)$,我们都需要求出对应的累积奖励$R(\tau)$。我们当然可以先把每一条从初始状态$s_0$到当前状态$s_t$所经历的轨迹对应的每一步奖励存储起来然后求出最终的累积奖励$R(\tau)$,即便 REINFORCE 算法在计算轨迹对应的价值方面做了一定的优化,但一个状态$s_t$背后的累积价值和梯度是需要通过包含很多个历史状态的轨迹来计算的,光是这样想恐怕读者们眼泪都要止不住掉下来。其次我们观察\eqref{eq:expect_policy_2},我们是要将所有轨迹的概率和对应的累积奖励相乘然后累加得到最终的价值期望,也就是目标函数。那么这里所有的轨迹是哪些轨迹呢?怎么数出来的?可能有些读者会说我们可以先搜集一些轨迹来近似所有轨迹,那么到底要搜集多少条轨迹才能近似?当然你可以搜集成千上万条,但恐怕还没等你计算完这些轨迹的期望其他人可能都已经用 DQN 算法完美解决这个问题了。我们原本认为直接通过对策略的梯度进行优化会比以间接的方式先估计对应的价值然后选择动作收敛的速度要更快,如果计算轨迹期望都这么麻烦似乎失去了实现策略梯度算法的初衷。并且还有一个很重要的问题,如\figref{fig:traj_compute}所示(补一张类似于条条大路通罗马的可爱动漫图),实际过程中初始状态$s_0$到当前状态$s_t$的轨迹应该不止一条,我可能走一步就到了状态$s_t$,也可能走了很多步才到$s_t$,走一步和走很多步生成的轨迹对应的累积奖励多数情况下也是不一样的,这个时候我们是多个轨迹都算进去还是只考虑其中一条呢?

\begin{figure}[hbt]
    \centering
    \includegraphics[width=0.5\linewidth]{ch6/figs/expected_reward.png}
    \caption{轨迹的计算}
    \label{fig:traj_compute}
\end{figure}

这样绕来绕去始终得不到一个准确的答案,这个时候可能就会有读者灵光乍现,我们在马尔科夫决策过程中计算的奖励不应该是“长期奖励”吗?那么这时候恭喜你没有被调皮的作者绕进去!实际上,在所有的马尔可夫过程中我们优化的目标都是“长期”的价值期望,不明白的读者可以再温习一下马尔可夫决策过程的定义以及前面Q learning算法相关的推导内容,相信你一定能够回顾起来所谓“长期”的意义。暂时给一个小的结论,我们这里选择的轨迹和对应的累积价值都应该是长期的轨迹和价值,如此才能保证我们的目标函数计算出来的也是长期的总价值。换句话说,对于每条轨迹,从初始状态$s_0$到当前状态$s_t$所经历的步数应该是无限的,即$t \rightarrow \infty$。这时候读者可能会奇怪,有限步数就已经够难算了,无限步数还能算出来吗?答案是可以的,而且更简便快捷,在数学中就是这样,“无限的”往往比“有限的”更容易计算出来,这里就涉及到一个重要的概念,即马尔可夫链的平稳分布,还请跟随笔者的思路到下一小节详细了解什么是马尔可夫链的平稳分布。

\subsubsection{平稳分布}

在本小节中我们将会详细了解什么是马尔可夫链的平稳分布,先不急着抛出一堆抽象的概念和公式,我们先来看一个经典的例子。这个例子是这样的,社会学家在他们的研究中通常会把人按照经济状况分成三类:上层,中层和下层,这三层就代表着三种状态,我们分别用 1,2,3 来表示。并且社会学家还发现决定一个人经济阶层的最重要因素就是其父母的收入阶层,即如果一个人的经济阶层为上层,那么他的孩子会有 0.5 的概率继续处于上层,也会有 0.4 的概率变成中层,更有 0.1 的概率降到下层,当然这些概率数值只是笔者拍脑袋想出来以便于后面的计算的,并没有一定的统计依据。这些概率其实就是我们所说的马尔可夫链中的转移概率,同样对于其他经济阶层的人他们的孩子也会有一定的概率变成上、中、下的任一经济阶层,如\figref{fig:markov_economy}所示。

\begin{figure}[hbt]
    \centering
    \includegraphics[width=0.5\linewidth]{ch6/figs/markov_economy.png}
    \caption{期望的奖励}
    \label{fig:markov_economy}
\end{figure}

这样我们就可以列出转移概率矩阵,如\eqref{eq:tran_prob_economy}所示。

\begin{equation}
    \label{eq:tran_prob_economy}
    P=\left[\begin{array}{lll}
    0.5 & 0.4 & 0.1 \\
    0.2 & 0.6 & 0.2 \\
    0.05 & 0.45 & 0.5
    \end{array}\right]
\end{equation}

我们假设有这么一批数量足够的人,称之为第 1 代人,他们的经济阶层比例为$\pi_0=[0.15,0.62,0.23]$,那么根据上面的转移概率矩阵我们就可以求出第二代的阶层比例。怎么求呢?首先求出第二代上层的比例,我们知道第一代人中有 0.15 的比例是上层,这 0.15 比例的人中子代为上层的概率是 0.5, 而第一代人中 0.62 比例的中层会有 0.2 的概率流入上层,0.23 比例的下层中其子代也会有 0.05 的概率流入上层,那么最后第二代上层的比例就为 $0.15 \times 0.5 + 0.62 \times 0.2 + 0.23 \times 0.05 = 0.2105 \approx 0.210$,依次类推,第二代中层的比例为$0.15 \times 0.4 + 0.62 \times 0.6 + 0.23 \times 0.45 \approx 0.536$,第二代下层的比例为$0.254$,这样我们就能得出第二代的阶层比例为$\pi_1=[0.210,0.536,0.254]$。这里细心的读者会发现不需要这么麻烦的计算过程,只要学过线性代数利用矩阵向量相乘就能得到,即$\pi_1 = \pi_0 P = [0.210,0.536,0.254]$。同理,第二代人的比例也可以求出,即 $\pi_2 = \pi_1 P = \pi_0 P^2$,依次类推,第n代人的比例为$\pi_n = \pi_0 P^n$。既然这本书同时也是教大家如何代码实战的,这里我们 Python 代码来求出前 10 代人的比例,如下。


\begin{lstlisting}[language=Python]
import numpy as np
pi_0 = np.array([[0.15,0.62,0.23]])
P = np.array([[0.5,0.4,0.1],[0.2,0.6,0.2],[0.05,0.45,0.5]])
for i in range(1,10+1):
    pi_0 = pi_0.dot(P)
    print(f"第{i}代人的比例为:")
    print(np.around(pi_0,3))
\end{lstlisting}

我们可以很快获得计算的结果,如下。

\begin{lstlisting}[language=Bash]
第1代人的比例为:
[[0.211 0.536 0.254]]
第2代人的比例为:
[[0.225 0.52  0.255]]
第3代人的比例为:
[[0.229 0.517 0.254]]
第4代人的比例为:
[[0.231 0.516 0.253]]
第5代人的比例为:
[[0.231 0.516 0.253]]
第6代人的比例为:
[[0.231 0.516 0.253]]
第7代人的比例为:
[[0.232 0.516 0.253]]
第8代人的比例为:
[[0.232 0.516 0.253]]
第9代人的比例为:
[[0.232 0.516 0.253]]
第10代人的比例为:
[[0.232 0.516 0.253]]
\end{lstlisting}

这里忽略程序算出来的小数取舍问题,比如第一代人我们手算的比例而$[0.210,0.536,0.254]$,而程序却是$[0.211,0.536,0.254]$,这是因为程序内置保留有效数字的规则问题,不影响整体结果。回归正题,从上面的结果中,我们发现从第5代开始经济阶层的比例开始神奇地固定了下来。换句话说,无论初始状态是什么,经过多次概率转移之后都会存在一个稳定的状态分布。其次我们只需要知道这个稳定的分布并乘以对应的价值,就可以计算所谓的长期收益了。

现在我们可以正式地总结一下马尔可夫链的平稳分布了,对于任意马尔可夫链,如果满足以下两个条件:

\begin{itemize}
    \item 非周期性:由于马尔可夫链需要收敛,那么就一定不能是周期性的,实际上我们处理的问题基本上都是非周期性的,这点不需要做过多的考虑。
    \item 状态连通性:即存在概率转移矩阵$P$,能够使得任意状态$s_0$经过有限次转移到达状态$s$,反之亦然。
\end{itemize}

这样我们就可以得出结论,即该马氏链一定存在一个平稳分布,我们用$d^{\pi}(s)$表示,可得到\eqref{eq:markov_station}:

\begin{equation}
    \label{eq:markov_station}
    d^\pi(s)=\lim _{t \rightarrow \infty} P\left(s_t=s \mid s_0, \pi_\theta\right)
\end{equation}

我们回顾前面小节中计算轨迹概率的公式$P_{\theta}(\tau)$,可以发现如果轨迹$\tau$的初始状态是$s_0$并且终止状态是$s$的话,轨迹概率公式$P_{\theta}(\tau)$跟平稳分布的$d^\pi(s)$是等效的,当然前提是该条轨迹必须“无限长”,即$t \rightarrow \infty$。但是平稳分布与轨迹概率公式相比,它的好处就是只涉及一个定量即初始状态$s_0$和一个变量$s$。对于每个状态$s$,我们用$V^{\pi}(s)$表示策略$\pi$下对应的价值,读者们现在可以往前回顾,为什么笔者说策略梯度算法跟基于价值函数的算法都是在计算累积状态的价值期望了,此时策略梯度算法目标函数就可以表示为\eqref{eq:pg_station_object}。

\begin{equation}
    \label{eq:pg_station_object}
    J(\theta)=\sum_{s \in \mathcal{S}} d^\pi(s) V^\pi(s)=\sum_{s \in \mathcal{S}} d^\pi(s) \sum_{a \in \mathcal{A}} \pi_\theta(a \mid s) Q^\pi(s, a)
\end{equation}

同样可以利用对数微分技巧求得对应的梯度,如\eqref{eq:pg_station_object_grad}。

\begin{equation}
    \label{eq:pg_station_object_grad}
    \begin{aligned}
    \nabla_\theta J(\theta) & \propto \sum_{s \in \mathcal{S}} d^\pi(s) \sum_{a \in \mathcal{A}} Q^\pi(s, a) \nabla_\theta \pi_\theta(a \mid s) \\
    &=\sum_{s \in \mathcal{S}} d^\pi(s) \sum_{a \in \mathcal{A}} \pi_\theta(a \mid s) Q^\pi(s, a) \frac{\nabla_\theta \pi_\theta(a \mid s)}{\pi_\theta(a \mid s)} \\
    &=\mathbb{E}_\pi\left[Q^\pi(s, a) \nabla_\theta \log \pi_\theta(a \mid s)\right]
    \end{aligned}
\end{equation}

可以发现该梯度跟前面小节求出的\eqref{eq:pg_ob_grad_2}的形式是类似的,只是变量从轨迹$\tau$换成了状态$s$,更便于实际问题的求解。在本书前面所讲的值迭代算法中,我们优化的目标是所有状态对应的$V(s)$,值迭代算法解决的问题是马尔可夫奖励过程。而强化学习的基本问题是马氏决策过程,因此在 Q learning 或 DQN 算法中,我们优化的是所有状态对应的 Q 值$Q^\pi(s, a)$,其中$Q^\pi(s, a)=\sum_{a \in \mathcal{A}} \pi(a \mid s) Q^\pi(s, a)$,优化所有的 Q 值之后再使用$\varepsilon - greedy$之类的策略选择动作。而在策略梯度算法中,我们是同时优化策略部分$\nabla_\theta \log \pi_\theta(a \mid s)$和价值部分$Q^\pi(s, a)$的,其中策略部分我们一般叫做 Actor ,价值部分叫做 Critic 。到这里就已经不是简单的纯策略梯度算法了,而是同时结合了基于价值和策略梯度的算法,我们一般把这类算法称之为 Actor-Critic 算法。这里 Actor 或者说策略梯度算法相比于基于价值的算法,其最大的好处就是同时可以适用于离散动作和连续动作空间的环境,而仅仅基于价值的算法比如 DQN 等算法只能适用于离散动作空间的问题。此外基于价值的算法只能通过众多动作价值中选择出一个最大价值对应的动作,是确定性的,而策略梯度算法中的 Actor 可以用一些随机分布比如高斯分布来表示,即能够使用随机策略。而有些实际问题的最优策略恰恰是随机策略,这种情况下基于价值的算法也无法解决。而这里的 Critic 使用了 DQN 算法中的 Q 值来表示,但是也会有更好的表示方法,这点我们将在后续讲解 A2C 和 GAE 等算法的章节中详细展开。我们将在下一小节中先简要介绍一下 Actor 的常用设计方式。

\subsubsection{策略函数的设计}

这一小节中我们将简要讲述 Actor 的常用设计方式,也就是策略函数$\pi_\theta(a \mid s)$。 对于离散动作空间的问题,最常用的策略函数就是 softmax 函数,softmax 函数在深度学习中通常作为最后一层网络用于多分类,而在强化学习中则使用描述状态和行为的特征函数$\phi(s,a)$和参数$\theta$的线性组合来权衡一个动作发生的概率,如\eqref{eq:softmax_act}:


\begin{equation}
    \label{eq:softmax_act}
    \pi_\theta(s, a)=\frac{e^{\phi(s, a)^T} \theta}{\sum_b e^{\phi(s, b)^T}}
\end{equation}

对应的梯度也可方便求得,如\eqref{eq:softmax_act_grad}。

\begin{equation}
    \label{eq:softmax_act_grad}
    \nabla_\theta \log \pi_\theta(s \mid a)=\phi(s, a)-\mathbb{E}_{\pi_\theta}[\phi(s, .)]
\end{equation}

而对于连续动作空间,通常策略对应的动作可以从高斯分布${\mathbb{N}}\left(\phi(s)^{\mathbb{T}} \theta, \sigma^2\right)$,对应的梯度也可求得:
\begin{equation}
    \nabla_\theta \log \pi_\theta(s, a)==\frac{\left(a-\phi(s)^T \theta\right) \phi(s)}{\sigma^2}
\end{equation}

\subsection{关键词}

策略(policy):在每一个演员中会有对应的策略,这个策略决定了演员的后续动作。具体来说,策略就是对于外界的输入,输出演员现在应该要执行的动作。一般地,我们将策略写成 $\pi$ 。

回报(return):一个回合(episode)或者试验(trial)得到的所有奖励的总和,也被人们称为总奖励(total reward)。一般地,我们用 $R$ 来表示它。

轨迹(trajectory):一个试验中我们将环境输出的状态 $s$ 与演员输出的动作 $a$ 全部组合起来形成的集合称为轨迹,即 $\tau=\left\{s_{1}, a_{1}, s_{2}, a_{2}, \cdots, s_{t}, a_{t}\right\}$ 。

奖励函数(reward function):用于反映在某一个状态采取某一个动作可以得到的奖励分数,这是一个函数。即给定一个状态-动作对 ($s_1$,$a_1$) ,奖励函数可以输出 $r_1$ 。给定 ($s_2$,$a_2$),它可以输出 $r_2$。 把所有的 $r$ 都加起来,我们就得到了 $R(\tau)$ ,它代表某一个轨迹 $\tau$ 的奖励。

期望奖励(expected reward):$\bar{R}_{\theta}=\sum_{\tau} R(\tau) p_{\theta}(\tau)=E_{\tau \sim p_{\theta}(\tau)}[R(\tau)]$。

REINFORCE:基于策略梯度的强化学习的经典算法,其采用回合更新的模式。


\subsection{习题}

\kw{4-1} 如果我们想让机器人自己玩视频游戏,那么强化学习中的3个组成部分(演员、环境、奖励函数)具体分别代表什么?

\kw{4-2} 在一个过程中,一个具体的轨迹{$s_1 , a_1 , s_2 , a_2$}出现的概率取决于什么?

\kw{4-3} 当我们最大化期望奖励时,应该使用什么方法?

\kw{4-4} 我们应该如何理解策略梯度的公式呢?

\kw{4-5} 我们可以使用哪些方法来进行梯度提升的计算?

\kw{4-6} 进行基于策略梯度的优化的技巧有哪些?

\kw{4-7} 对于策略梯度的两种方法,蒙特卡洛强化学习和时序差分强化学习两种方法有什么联系和区别?

\kw{4-8} 请详细描述REINFORCE算法的计算过程。


\subsection{面试题}

\kw{4-1} 友善的面试官:同学来吧,给我手动推导一下策略梯度公式的计算过程。

\kw{4-2} 友善的面试官:可以说一下你所了解的基于策略梯度优化的技巧吗?
\subsection{本章小结}

在推导策略梯度算法目标函数的梯度过程中,我们用到了一个很常用的对数微分技巧,请大家务必熟练掌握。在基础推导的部分中我们最后推导出了以轨迹为基础的策略梯度公式,尽管 REINFORCE 算法在此基础上做了一定的优化,但是结合公式和实际的算法实践可以看出 REINFORCE 算法不仅计算繁琐,而且收敛困难。最后我们介绍策略梯度公式的进阶推导,将复杂的轨迹计算转变成了简单的状态计算,并且引出了基于价值和策略梯度结合的一类算法,即 Actor - Critic 算法,关于此类算法将会在后续章节中详细展开。

\bibliographystyle{gbt7714-numerical}
\bibliography{ref.bib}

% \section*{参考文献}
% \begin{itemize}
%     \item \href{https://github.com/zhoubolei/introRL}{Intro to Reinforcement Learning (强化学习纲要)}
%     \item \href{https://nndl.github.io/}{神经网络与深度学习}
%     \item \href{https://book.douban.com/subject/35043939/}{百面深度学习}
% \end{itemize}




    % 
\section{ Actor-Critic 算法}
\subsection{ Q Actor-Critic 算法}
\subsection{ A2C 算法}
\subsection{ A3C 算法}
\subsection{ GAE 算法}

\subsection{实战:A2C 算法}
\subsection{实战:多线程与多进程}
\subsection{实战:A3C 算法}
\subsection{实战:GAE 算法}


\subsection{关键词}
\subsection{习题}
\subsection{面试题}
\subsection{本章小结}
    % 
\section{ DDPG 算法}
\subsection{ DPG 算法}
\subsection{ DDPG 算法}
\subsection{实战:DDPG算法}

简单引出最大熵强化学习

\subsection{关键词}
\subsection{习题}
\subsection{面试题}
\subsection{本章小结}
    % 
\section{ PPO 算法}
\subsection{ TRPO 算法}
\subsection{ PPO 算法}
\subsection{实战:PPO算法}


\subsection{关键词}
\subsection{习题}
\subsection{面试题}
\subsection{本章小结}
    % 
\section{ SAC 算法}
\subsection{最大熵强化学习}
\subsection{ Soft Q 算法}
\subsection{ SAC 算法}
\subsection{实战:Soft Q 算法}
\subsection{实战:SAC 算法}

\subsection{关键词}
\subsection{习题}
\subsection{面试题}
\subsection{本章小结}
    % 
\section{强化学习进阶}
\subsection{探索策略}
\subsection{逆强化学习}
\subsubsection{模仿学习}
\subsubsection{行为克隆}
\subsubsection{最大熵逆强化学习}
\subsection{分层强化学习}
\subsubsection{稀疏奖励}
\subsubsection{好奇心模块}
\subsubsection{课程学习}
\subsection{离线强化学习}
\subsection{多智能体强化学习}
\subsubsection{ QMIX 算法}
\subsubsection{ MADDPG 算法}
\subsubsection{ MAPPO 算法}

\subsection{关键词}
\subsection{习题}
\subsection{面试题}
\subsection{本章小结}
    % \section{经典论文解读}
\subsection{AlphaStar}
\subsection{AlphaZero}
\subsection{AlphaFold}
    % \appendix
\section{附录A习题解答}
\section{附录B面试题解答}

\end{document}
